\section{Discussion}
\label{sec:discussion}

\subsection{The period gap}
\label{sec:period_gap}

The origin of the rotation period gap, first identified
by \citet{mcquillan2013} and visible in figures \ref{fig:age_cut} and
\ref{fig:dispersion_period_teff} still remains a mystery.
This gap can be seen as an under-density of points between the 0.7-1.0 and
1.0-1.5 Gyr age ranges in figure \ref{fig:age_cut} and roughly follows a line
of constant gyrochronal age of around 1.1 Gyr \citep[according to the
gyrochronology relation of][]{angus2019}, as shown in figure
\ref{fig:dispersion_period_teff}.
Several explanations for the gap's origin have been proposed, including a
discontinuous star formation history \citep{mcquillan2013, davenport2017,
davenport2018}, a rapid change in magnetic field structure
\citep{reinhold2019}, and erroneous rotation period measurements that are
incorrect by a factor of two \citep{koen2018}.
The latter explanation can be ruled out because stars below the gap have
smaller velocity dispersions than the stars above the gap, indicating that
they are kinematically younger \citep{mcquillan2013, davenport2018}, as
evident in figures \ref{fig:age_cut} and \ref{fig:dispersion_period_teff}.

For stars below the gap, in the 0.7-1.0 Gyr age range shown in figure
\ref{fig:age_cut}, velocity dispersion is relatively constant as a function of
temperature, however above the gap, in the 1.0-1.5 Gyr age range and older,
velocity dispersion increases with \teff.
The coolest stars in the 1.0-1.5 Gyr age range have the same velocity
dispersion as the mid-temperature stars in the age range above, and the
hottest stars in the age range above {\it that}, which indicates that the
period-\teff\ relations are {\it flat} at rotation periods between $sim$15-25
days, for 5500 K $<$ \teff\ $<$ 4000 K.
Below the gap, velocity dispersion within a given period range appears to {\it
decrease} with decreasing temperature.
The opposite appears to be true above the gap.

The gap may be positioned at a significant Rossby number/age at which stellar
magnetic dynamos go through a transition.
Perhaps before the age of 1.1 Gyr, or at Rossby numbers less than ???,
magnetic braking is more efficient for stars with deeper convection zones.
Once stars reach this critical age or Rossby number their magnetic fields
undergo some transition, which produces the gap in the rotation period-\teff\
plane and after this transition, magnetic braking efficiency no longer
increases with decreasing mass.
Of course, it may be a coincidence that the gyrochronology relations seem to
only flatten off above the period gap and we lack a sufficient quantity of
data to do more than speculate here.
New rotation periods from the \ktwo\ and \tess\ missions may be able to
validate or rule out this hypothesis in the future.

In the $\sim$ 1.1 Gyr NGC 6811 cluster, the rotation periods of mid-K dwarfs
are faster than expected; their rotational evolution appears to have stalled,
and they have similar rotation periods to the 650 Myr Praesepe cluster
\citep{curtis2019}.
The rotation periods of the K dwarfs in NGC 6811 are plotted in figure
\ref{fig:dispersion_period_teff}.
Although NGC 6811's G dwarfs fall on the 1.1 Gyr gyrochronology model, the K
dwarfs lie only a little above the 0.65 Gyr gyrochronology model.
NGC 6811 straddles the rotation period gap: its G dwarfs lie above it and its
K dwarfs lie below it.
This cluster may be the `missing link' that connects two epoch of stellar
spin-down.
However, figure \ref{fig:age_cut} indicates that the Praesepe gyrochronology
model is, on average, a {\it good} model for stars younger than $\sim$1 Gyr
% and this model has a very different shape to the NGC 6811 cluster.
suggesting that stars in the field follow the period-color/\teff\ relation of
Praesepe, at least up to around 1 Gyr.
We do not see strong evidence of K dwarf stalling in the field.
% , however without more observations of middle-aged open clusters it : an
% early stage where the period-\teff\ relation for cool dwarfs has a negative
% slope and a late stage where it has a positive slope.
% In this case the period gap may delineate the transition between these two
% regimes and is the point at which stellar magnetic dynamos likely undergo a
% dramatic structural shift at an age of $\sim$ 1.1 Gyr.

% \subsection{Caveats}
% \label{sec:caveats}

% There are some alternative explanations for the trends seen in figures
% \ref{fig:age_cut} and \ref{fig:}
% We list these potential issues below.
% \begin{itemize}
% \item{
% The selection function for these data is extremely complicated, etc.
% }
% \item{
% Redenning for the coolest stars.
% }
% \item{
% Planets/companions for the coolest stars.
% }
% \item{
% Subgiant contamination/weakened magnetic braking.
% }
% \end{itemize}
