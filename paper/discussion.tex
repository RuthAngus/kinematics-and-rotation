\section{Discussion}
\label{sec:discussion}

% \subsection{The period gap}
% \label{sec:period_gap}

The results presented above indicate that stars of spectral type ranging from
late G to late K follow a braking law that changes over time.
In particular, the relationship between rotation period and effective
temperature appears to flatten out and eventually invert.
These results provide further evidence for `stalled' rotational evolution of K
dwarfs, like that observed in open clusters \citep{curtis2019} and reproduced
by models that vary angular momentum transport between stellar core and
envelope with time and mass \citep{spada2019}.

The velocity dispersions of stars in the \mct\ sample provide the following
picture of rotational evolution.
At young ages, stellar rotation period decreases with mass, likely because
lower-mass stars with deeper convection zones have stronger magnetic fields,
larger Alfv\'en radii and therefore experience greater angular momentum loss
rate.
According to the \citet{spada2019} models, there is minimal transportation of
angular momentum from the surface to the core of the star at these young ages,
so the surface slows down but the core keeps spinning rapidly.
At intermediate ages, rotation period is constant with mass, and at late ages
rotation period {\it increases} with mass for GK dwarfs.
The explanation for this, according to the \citet{spada2019} models, is that
lower-mass stars are still braking more efficiently at these intermediate and
old ages but their cores are more tightly coupled to their envelopes, allowing
angular momentum transport between the two interior layers.
Angular momentum resurfaces and prevents the stellar envelopes from
spinning-down rapidly.

% The origin of the rotation period gap, first identified by
% \citet{mcquillan2013} and visible in figures \ref{fig:age_cut} and
% \ref{fig:dispersion_period_teff} still remains a mystery.
% This gap can be seen as an under-density of points between the 0.7-1.0 and
% 1.0-1.5 Gyr age ranges in figure \ref{fig:age_cut} and roughly follows a line
% of constant gyrochronal age of around 1.1 Gyr \citep[according to the
% gyrochronology relation of][]{angus2019}, as shown in figure
% \ref{fig:dispersion_period_teff}.
% Several explanations for the gap's origin have been proposed, including a
% discontinuous star formation history \citep{mcquillan2013, davenport2017,
% davenport2018} and a change in magnetic field structure causing a brief period
% where rotational variability is reduced and rotation periods cannot be
% measured \citep{reinhold2019}.
% Our results indicate there might be a slight difference in rotational
% evolution below and above the gap.
% Stars below the gap appear to be in the regime where wind-braking is dominant
% (rotation period decreases with increasing mass at a given age) and stars
% above the gap are in a regime where rotational coupling is dominant (rotation
% period increases with increasing mass at a given age).
% This can be seen in figures \ref{fig:age_cut} and \ref{fig:vplot}.
% For stars below the gap, in the 0.7-1.0 Gyr age range shown in figure
% \ref{fig:age_cut}, velocity dispersion is relatively constant as a function of
% temperature, however above the gap, in the 1.0-1.5 Gyr age range and older,
% velocity dispersion increases with \teff.
% In figure \ref{fig:age_cut}, velocity dispersion within a given period range
% appears to {\it decrease} with decreasing temperature below the gap.
% The opposite appears to be true above the gap.
% This suggests that perhaps stars transition from a wind braking regime to a
% rotational coupling regime as they pass across the gap.
% However, it is unclear how this process could {\it create} the gap.
% A gap could be created if stars rapidly lose angular momentum over a short
% period of time, leading to a dearth of stars in the gap.
% However, a sudden redistribution of angular momentum from the core to the
% surface, should cause stellar rotation periods to {\it decrease}, not
% increase.
% In addition, this theory is not supported by observations of the NGC 6811
% cluster whose K dwarf members have already undergone stalled braking below and
% before they reach the gap.
% The gap may be associated with a wind braking-rotational coupling transition,
% however our results only provide weak evidence for this, so we can only
% speculate until more data become available.

% The \citet{spada2019} models do predict that stars transition from a regime
% where wind-braking is dominant to a regime where rotational coupling is
% domimant at different times for different masses.
% However, these models are based on observations of NGC 6811, whose K dwarfs
% already show signs of stalling by 1.1 Gyr.
% For stellar masses of 0.7 M$_\odot$, this transition takes place at $\sim$ 1
% Gyr.
% It happens at 500 Myr for 0.9 M$_\odot$ and 5 Gyr for 0.5 M$_\odot$
% \citep[see][for details]{spada2019}.
% We propose a new theory for the origin of the rotation gap as the transition
% between the wind-braking and rotational coupling dominated regimes.
% The gap could be caused by rapidly transitioning across the gap
% as

% These results are extremely tentative and should be interpreted with caution:
% more data are needed to confirm this behavior.

% For stars below the gap, in the 0.7-1.0 Gyr age range shown in figure
% \ref{fig:age_cut}, velocity dispersion is relatively constant as a function of
% temperature, however above the gap, in the 1.0-1.5 Gyr age range and older,
% velocity dispersion increases with \teff.
% The coolest stars in the 1.0-1.5 Gyr age range have the same velocity
% dispersion as the mid-temperature stars in the age range above, and the
% hottest stars in the age range above {\it that}, which indicates that the
% period-\teff\ relations are {\it flat} at rotation periods between $\sim$15-25
% days, for 5500 K $<$ \teff\ $<$ 4000 K.
% Below the gap, velocity dispersion within a given period range appears to {\it
% decrease} with decreasing temperature.
% The opposite appears to be true above the gap.

% The gap may be positioned at a significant Rossby number/age at which stellar
% magnetic dynamos go through a transition.
% Perhaps before the age of 1.1 Gyr, or at Rossby numbers less than ???,
% magnetic braking is more efficient for stars with deeper convection zones.
% Once stars reach this critical age or Rossby number their magnetic fields
% undergo some transition, which produces the gap in the rotation period-\teff\
% plane and after this transition, magnetic braking efficiency no longer
% increases with decreasing mass.
% Of course, it may be a coincidence that the gyrochronology relations seem to
% only flatten off above the period gap and we lack a sufficient quantity of
% data to do more than speculate here.
% New rotation periods from the \ktwo\ and \tess\ missions may be able to
% validate or rule out this hypothesis in the future.

% In the $\sim$ 1.1 Gyr NGC 6811 cluster, the rotation periods of mid-K dwarfs
% are faster than expected; their rotational evolution appears to have stalled,
% and they have similar rotation periods to the 650 Myr Praesepe cluster
% \citep{curtis2019}.
% The rotation periods of the K dwarfs in NGC 6811 are plotted in figure
% \ref{fig:dispersion_period_teff}.
% Although NGC 6811's G dwarfs fall on the 1.1 Gyr gyrochronology model, the K
% dwarfs lie only a little above the 0.65 Gyr gyrochronology model.
% NGC 6811 straddles the rotation period gap: its G dwarfs lie above it and its
% K dwarfs lie below it.
% This cluster may be the `missing link' that connects two epoch of stellar
% spin-down.
% However, figure \ref{fig:age_cut} indicates that the Praesepe gyrochronology
% model is, on average, a {\it good} model for stars younger than $\sim$1 Gyr
% % and this model has a very different shape to the NGC 6811 cluster.
% suggesting that stars in the field follow the period-color/\teff\ relation of
% Praesepe, at least up to around 1 Gyr.
% We do not see strong evidence of K dwarf stalling in the field.
% % , however without more observations of middle-aged open clusters it : an
% % early stage where the period-\teff\ relation for cool dwarfs has a negative
% % slope and a late stage where it has a positive slope.
% % In this case the period gap may delineate the transition between these two
% % regimes and is the point at which stellar magnetic dynamos likely undergo a
% % dramatic structural shift at an age of $\sim$ 1.1 Gyr.

% \subsection{Caveats}
% \label{sec:caveats}

% There are some alternative explanations for the trends seen in figures
% \ref{fig:age_cut} and \ref{fig:}
% We list these potential issues below.
% \begin{itemize}
% \item{
% The selection function for these data is extremely complicated, etc.
% }
% \item{
% Redenning for the coolest stars.
% }
% \item{
% Planets/companions for the coolest stars.
% }
% \item{
% Subgiant contamination/weakened magnetic braking.
% }
% \end{itemize}
