% \section{Discussion}
% \label{sec:discussion}

% Aside from the assumption that mass-dependent dynamical heating is {\it not}
% the main cause of the increased velocity dispersion for cooler stars, a number
% of assumptions went into this analysis which should be addressed before a
% strong conclusion about the nature of stellar spin down is drawn.
% There are some potential confounders that could produce trends between
% velocity dispersion and effective temperature, of which the main ones are
% listed below.
% \begin{itemize}

    % \item{{\bf Extinction.}
% Although we dereddened stars before their ages were estimated, if the
    %     reddening values were underestimated, the gyrochronal ages of stars
    %     would be under-estimated.
% This effect would be largest for the late Ks and early Ms, where the shape of
    %     the period-color relation is more steeply sloped, and stars can more
    %     easily move into the wrong age bin with a small perturbation in color
    %     or temperature.
% Old stars would be incorrectly assigned young ages, and this would happen more
    %     often for cooler stars.
% This effect could lead to an increase in velocity dispersion with \teff, even
% if the gyrochronology relation was perfect.}

    % \item{{\bf Mass-dependent orbital heating.}
% A key assumption going into this analysis is that the orbits of all stars are
% heated at the same rate, regardless of their mass.
% However, if the heating mechanism leads to preferential heating of lower mass
% stars, those stars would have a larger velocity dispersion, not caused by
% ageing.
% No strong evidence has yet been provided to show a mass-dependence in orbital
% heating, however this is an extremely difficult exercise because mass and age
% are so strongly correlated.
% The mass-age degeneracy is reduced in the lowest mass stars which represent
    %     the initial mass function of the Solar neighborhood.
% No mass-dependent heating was found by \citet{faherty2009} who looked
% at the velocity dispersions of low-mass stars.
% Another argument against the increased velocity dispersion as a function of
    %     \teff\ being caused by mass-dependent orbital heating, is that heating
    %     acts most strongly at young ages and most weakly at old ages, \ie\ on
    %     a logarithmic timescale.
% This is related to the fact that stars are more difficult to scatter once they
% are on orbits with a large out-of-plane component and they spend more time out
% of the mid-plane.
% Our data show the opposite trend: stars show little mass-dependent heating
    %     before around 1 Gyr.
% Moreover, the stars in the Geneva-Copenhagen study, used to calibrate the AVR
    %     \citep{holmberg2009} are F and G stars, around 1-3 times as massive as
    %     the K stars in our sample.
% % nordstrom2004, jorgensen2005,
% The ages for our sample of K stars, predicted by gyrochronology agree with the
    %     AVR calibrated using F and G stars, which implies that either
    %     mass-dependent heating is a weak effect, or, mass-dependent heating is
    %     strong and, by coincidence, gyrochronology underpredicts the ages of
    %     the early K dwarfs by just the right amount that they still appear to
    %     agree with the \citet{holmberg2009} AVR.
% We therefore expect that the increased velocity dispersion at cooler
    %     temperatures is mostly caused by incorrect age-grouping, \ie\ the
    %     period-color relation is incorrect for old, low-mass stars, and that
    %     any mass-dependent heating, while it may contribute at a low level to
    %     this result, is not the dominant cause.
% }

    % \item{{\bf Selection bias.}
% The \gaia\
% catalog is incompletebecause they are too faint.
% This is simply due to the fact that \kepler's field of view was aimed at low
    %     galactic latitudes, so nearby \kepler\ stars are close to the galactic
    %     plane.
% The heights of stars above the galactic plane, and correspondingly their
    %     velocities and ages, are therefore correlated with distance.
% Almost no high velocity stars appear in our sample at temperatures cooler than
% 3500, which is why the coolest stars were excluded from our sample, however
% the selection function could still affect stars hotter than 3500 K, and is
% suggested by the decrease in velocity dispersions for cool stars of all ages
% in figure \ref{fig:age_cut}.
% However, this selection effect would act in opposition to the observed trend.
% }

% \item{{\bf Weakened magnetic braking.}
% The rotation periods of old stars are faster than predicted by Skumanich-like
% magnetic braking \citep{angus2015, vansaders2016, metcalfe2019}.
% Magnetic braking is expected to become inefficient once stars approach a
% Rossby number (the ratio of rotation period to convective turnover time) of
% around 2 \citep{vansaders2016, vansaders2018}.
% Since hot stars have shallow convection zones and shorter overturn timescales,
% they reach $Ro = 2$ and stop spinning down at shorter rotation periods than
% cool stars.
% For this reason, some of the hotter stars in the sample may already have
% stopped spinning down, and could be much older than a traditional
% gyrochronology relation (like the simple Praesepe-based relation used in this
% analysis) would suggest.
% However, this effect would work to cancel out the trend observed in figure
% \ref{fig:age_cut} because hotter groups of stars would contain more old stars
% with large velocities.
% }

% \item{Stellar or exoplanet companions.}
% Hotter stars are more likely to be binaries.
% This would make synchronized some stars seem younger, more likely for massive
% stars, although really only likely at short rotation periods.

\end{itemize}
