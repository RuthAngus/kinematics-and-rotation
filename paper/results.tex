% word limit: 500
\section{Results and Discussion}
\label{sec:results}

% \subsection{Velocity dispersion of coeval groups}
% \label{sec:age_cut}

% To explore the relationship between rotation period, \teff\ and age for field
% K dwarfs, we calculated gyrochronal ages using dereddened \Gaia\ photometry
% ($G$, $G_{BP}$ and $G_{RP}$) and rotation periods from \citet{mcquillan2014}.
% We used the \citet{angus2019} gyrochronology relation which is a simple,
% separable relation in \gcolor\ color and age, calibrated using the
% period-color relation of Praesepe and the period-age relation of Praesepe and
% the Sun.
% The large number of Praesepe members with precise rotation periods from the
% \ktwo\ mission \citep{howell2014, douglas2017, rebull2017}, spanning spectral
% types F through early M, makes it a good cluster for calibrating the
% period-color relation of stars at young ages.
% However, this is an extremely simple model and although it accurately
% describes the rotation periods of F and G stars up to around 2.5 Gyr (the age
% of NGC 6819 -- the oldest cluster with available rotation periods), it
% over-predicts the rotation periods of K dwarfs in the 1.1 Gyr NGC 6811
% cluster.

% We then selected groups of stars within different gyrochronal age ranges and
% calculated the standard deviation of \vb\ velocities (\sigmavb), as a function
% of effective temperature for each age group.
% Ages were calculated using dereddened \gaia\ \gcolor\ color, however
% throughout this paper we show rotation periods as a function of effective
% temperature, \teff, because it is easier to divide stars into bins of roughly
% equal numbers in \teff-space than in color-space.

% \begin{figure}
%   \caption{
% Top: rotation period vs effective temperature for stars in the \mct\
%     catalog.
%     The full catalog, with subgiants and visual binaries removed is shown in
%     grey, and stars selected to be in different age groups (between \tmin\ and
%     \tmax\ K) are overlayed in color.
% These age groups were selected using the \citet{angus2019} gyrochronology
%     relation.
% The legend in the center of the figure lists the age range, in Gyr, of each
%     group.
% Bottom: velocity dispersion vs effective temperature for each age
%     group.
% The color of the line corresponds to the color of the group shown in the top
%     panel.
% If the gyrochronal model were correct at all ages, and the stars in each group
%     were the same age across temperatures, the velocity dispersion would be
%     constant as a function of \teff.
% However, the velocity dispersions of the oldest age groups increase with
%     \teff, indicating the \citet{angus2019} gyrochronology model underpredicts
%     the the ages of late-K dwarfs relative to the ages of late G and early K
%     dwarfs at old ages.
% % An alternative explanation could be that the gyrochronology relation is
% %     correct and {\it mass-dependent heating} is responsible for the greater
% %     velocity dispersions of cooler stars.
% }
%   \centering
%     \includegraphics[width=1\textwidth]{age_cut}
% \label{fig:age_cut}
% \end{figure}
% The top panel of figure \ref{fig:age_cut} shows the full \mct\ sample
% (excluding visual binaries and subgiants) in grey, with coeval groups
% \citep[according to the][relation]{angus2019} shown in color.
% The color of the points corresponds to the age ranges specified in the legend
% (in Gyr), which also apply to the lines in the lower panel.
% The bottom panel shows the velocity dispersion, \sigmavb\ of each age group,
% as a function of effective temperature.
% % We only included stars within a temperature range of 5000 - 3500 K in our
% % analysis, as hotter stars are more likely to have stopped magnetic braking
% % \citep{vansaders2016}, which could bias the results.
% % Late M dwarfs were not included in our analysis because such faint stars
% % cannot be observed at large heights above the plane (because of the low
% % galactic latitude of the \kepler\ field, stars at high-Z are more distant),
% % which introduces a mass-dependent velocity bias: cooler populations of stars
% % are skewed towards lower velocity dispersions.
% % The coolest temperature bins in the lower panel of figure \ref{fig:age_cut}
% % have low velocity dispersions, indicating that this effect may already become
% % important at temperatures lower than $\sim$ 4000 K.
% Overall, figure \ref{fig:age_cut} shows that velocity dispersion increases
% with gyrochronal age across all temperatures, implying that both velocity
% dispersion and rotation period increases with age as expected.
% The flat velocity dispersion of young stars as a function of temperature
% shows that the Praesepe-calibrated gyrochronology relation accurately predicts
% the relative ages of {\it young} field stars.
% % To our knowledge, no gyrochronology relation has previously been demonstrated to
% % correctly predict ages (either relative or absolute) for such cool or such
% % young field stars.
% % This is not particularly remarkable however, since these young stars are a
% % similar age to the Praesepe cluster, which was used to calibrate the
% % period-color relation.
% % However, if the stars in each selected age group had the {\it same} age across
% % the temperature range, their velocity dispersion would be a constant function
% % of \teff.
% If \citet{angus2019} gyrochronology relation worked at all ages and
% temperatures, the bottom panel of figure \ref{fig:age_cut} would show a flat
% relationship between velocity dispersion and \teff at all ages.
% However, velocity dispersion {\it increases} as a function of temperature for
% old stars, meaning the \citet{angus2019} gyrochronology relation either
% under-predicts the ages of old, late-K dwarfs, or over-predicts the ages of
% old early-K and late-G dwarfs\footnote{Since \vb\ velocity dispersion only
% provides relative and not absolute ages, it is difficult to tell whether the
% ages of cool stars are being under-predicted, the ages of hot stars being are
% over-predicted, or both.}.
% This suggests that the relationship between rotation period and photometric
% color or \teff\ flattens out over time, and possibly even inverts.
% % The lines of constant age (isochrones) sweeping diagonally upwards in the top
% % panel of figure \ref{fig:age_cut} are too steeply sloped at old ages.

\subsection{The period-\teff\ relations, revealed}
\label{sec:the_reveal}

To explore the relationship between rotation period, \teff\ and velocity
dispersion, we first removed high and low velocity outliers from the \mct\
sample by performing 3$\sigma$ sigma-clipping on the \vb\ velocities.
Without sigma-clipping, we found that a small number of high velocity
outliers at the low-temperature end of our sample substantially raised the
velocity dispersion for cooler stars, however the overall trends remain the
same with, or without sigma-clipping.
We also limited the sample to temperatures in the range 5500 K $<$ \teff\ $<$
3500 K to avoid biases caused by the selection function at the faint, cool
end, and binarity or weakened braking \citep{vansaders2016} at the hot end.
Finally, we removed stars with short rotation periods by cutting out stars
with gyrochronal ages less than 0.5 Gyr.

The top panel of figure \ref{fig:vplot} shows rotation period versus effective
temperature for the \mct\ sample, coloured by the standard deviation of their
(\vb) velocities, where \sigmavb\ was calculated for groups of stars over a
grid in $\log_{10}$(period) and temperature.
If we assume that mass dependent heating does not strongly affect this sample
and \vb\ at low galactic latitudes is an unbiased tracer of \vz, then \vb\
velocity dispersion can be interpreted as an age proxy, and stars plotted in a
similar color in figure \ref{fig:vplot} are similar ages.
\begin{figure}
  \caption{
    Top: Rotation period vs effective temperature for stars in the \mct\
    sample, colored by the velocity dispersions of stars calculated over a
    grid in $\log_{10}$(period) and \teff.
    Black lines show isochrones from a gyrochronology model that projects the
    rotation-color relation of
    Praesepe to longer rotation periods over time \citep{angus2019}.
    These isochrones do not appear to reflect the evolution of field stars at
    long rotation periods/old ages: they do not trace lines of constant
    velocity dispersion.
    Isochrones are plotted at 0.5, 1, 1.5, 2, 2.5, 4 and 4.57 Gyrs in both top
    and bottom panels.
    Bottom: Same as top panel with rotation period vs {\it mass}
    \citep[from the Kepler Input Catalog][]{brown2011}.
    White lines show isochrones from a model that includes mass and
    age-dependent angular momentum transport between the core and envelope
    \citep{spada2019}.
    Qualitatively, these isochrones reflect the evolution of field
    stars at long rotation periods/old ages: they trace lines of constant
    velocity dispersion by reproducing periods of `stalled' rotational
    evolution for K-dwarfs.
}
  \centering
    \includegraphics[width=1\textwidth]{main_figure}
\label{fig:vplot}
\end{figure}
Overall, figure \ref{fig:vplot} shows that velocity dispersion increases with
rotation period across all temperatures, implying that both velocity
dispersion and rotation period increases with age as expected.
Black lines show isochrones from the \citep{angus2019} gyrochronology model,
which projects the rotation-color relation of Praesepe to longer rotation
periods over time.
These isochrones are plotted at 0.5, 1, 1.5, 2, 2.5, 4 and 4.57 Gyrs.
At the youngest ages, these isochrones seem to describe the data well: the
palest yellow (youngest) stars with the lowest velocity dispersions all fall
close to the 0.5 Gyr isochrone.
% Lines of constant age (isochrones) appear to follow the shape of the
% Praesepe-based gyrochronology model (black dotted line) at young ages.
% However, at older ages it appears that the relation between rotation period
% and \teff\ flattens out, until eventually rotation period {\it decreases} with
% decreasing effective temperature at a given age.
However, although the 0.5 Gyr and 1 Gyr isochrones do appear to trace constant
velocity dispersion/age among the field stars, by 1.5 Gyrs the isochrones
start to {\it cross} different velocity dispersion regimes.
For example, the 1.5 Gyr isochrone lies on top of stars with velocity
dispersions of around 10-11 kms$^{-1}$ at 5000-5500K and stars with $\sim$15
\kms\ velocity dispersions at 4000-4500K.
The isochrones older than 1.5 Gyr also cross a range of velocity dispersions.
If these were true isochrones however, they should follow lines of constant
velocity dispersion.
At ages older than around 1 Gyr, it appears that isochrones should have a more
flattened, or even inverted relationship between rotation period and effective
temperature than these Praesepe-based models.
% For late K-dwarfs older than around 1 Gyr, it appears that lower-mass stars
% rotate more rapidly than higher-mass stars.
% This is the opposite of the trend that is observed in young open clusters such
% as Praesepe and the Hyades.

The bottom panel of figure \ref{fig:vplot} shows velocity dispersion as a
function of rotation period and {\it mass}, obtained from the Kepler Input
Catalog \citep{brown2011}, with isochrones from the \citep{spada2019}
gyrochronology model shown as white lines.
These isochrones are also plotted at 0.5, 1, 1.5, 2, 2.5, 4 and 4.57 Gyrs.
Although these models do not trace lines of constant velocity dispersion at
0.5 and 1 Gyr, they do appear to reproduce trends in the data at ages of 1.5
Gyr and older.
The \citet{spada2019} models appear to qualitatively agree with the data:
their rotation period-\teff\ relation flattens out over time and eventually
inverts.

The results shown in figure \ref{fig:vplot} indicate that stars of spectral
type ranging from late G to late K follow a braking law that changes over
time.
In particular, the relationship between rotation period and effective
temperature appears to flatten out and eventually invert.
These results provide further evidence for `stalled' rotational evolution of K
dwarfs, like that observed in open clusters \citep{curtis2019} and reproduced
by models that vary angular momentum transport between stellar core and
envelope with time and mass \citep{spada2019}.
The velocity dispersions of stars in the \mct\ sample provide the following
picture of rotational evolution.
At young ages, stellar rotation period decreases with mass, likely because
lower-mass stars with deeper convection zones have stronger magnetic fields,
larger Alfv\'en radii and therefore experience greater angular momentum loss
rate.
According to the \citet{spada2019} model, there is minimal transportation of
angular momentum from the surface to the core of the star at these young ages,
so the surface slows down but the core keeps spinning rapidly.
At intermediate ages, rotation period is constant with mass, and at late ages
rotation period {\it increases} with mass for K-dwarfs.
The interpretation of this, according to the \citet{spada2019} model, is that
lower-mass stars are still braking more efficiently at these intermediate and
old ages but their cores are more tightly coupled to their envelopes, allowing
angular momentum transport between the two layers.
Angular momentum resurfaces and prevents the stellar envelopes from
spinning-down rapidly, and this effect is strongest for late K-dwarfs with
effective temperatures of $\sim$4000-4500K and masses $\sim$0.5-0.7 M$_\odot$.

It has been demonstrated that lower-mass stars remain magnetically active
longer than more massive stars, \citep{west2008, kiman2019}.
If the detectability of a rotation period is considered to be a magnetic
activity proxy, then our results provide further evidence for a mass-dependent
activity lifetime.
Figure \ref{fig:vplot} shows that the groups of stars with the largest
velocity dispersions are cooler than 4500 K.
This implies that the oldest stars with detectable rotation periods, are
cooler than 4500 K, \ie\ these-low mass stars stay active longer than more
massive stars.

\subsection{The period gap and synchronized binaries}
\label{sec:gap}

There is a sharp gap in the population of rotation periods, which lies just
above the 1 Gyr isochrone in the upper panel of figure \ref{fig:vplot}, whose
origin is unknown and is the subject of much speculation \citep{mcquillan2014,
davenport2018, reinhold2019}.
The Praesepe-based model appears to be valid below the gap but not above.
Although this may be coincidental (and more data would be needed to confirm a
connection) the gap may indeed separate a young regime where stellar cores are
decoupled from their envelopes from an old regime where these layers are more
tightly coupled.
If so, this could indicate that the phenomenon responsible for changing the
shape of isochrones in rotation-\teff\ space is related to the phenomenon that
produces the gap.

\begin{figure}
  \caption{
      Top: rotation period vs. effective temperate for stars in the \mct\
    sample, separated into three groups. Blue circles
      show stars with rotation periods longer than the
    period gap, orange squares show stars with rotation periods shorter than
    the gap, but longer than the lower edge of the main rotation period
    distribution, and green triangles show stars with rotation periods shorter
    than this lower edge.
    Stars were separated into these three groups using \citet{angus2019}
    gyrochronology models, with the scheme shown in the legend.
    Only stars cooler than 5000 K are plotted in
    the bottom panel in order to isolate populations above and below the
    period gap, which only extends up to temperatures of $\sim$4600 K.
    Bottom: the velocities of these groups of stars (in the direction of
    Galactic latitude, $b$) are shown as a function of rotation period.
}
  \centering
    \includegraphics[width=1\textwidth]{gap}
\label{fig:gap}
\end{figure}

Figure \ref{fig:gap} shows the velocity dispersions of stars in the \mct\
sample, with stars subdivided into three groups: those that rotate more
quickly than the major rotation period distribution (green triangles), those
with rotation periods shorter than the gap (orange squares), and those with
rotation periods longer than the gap (blue circles).
Stars were separated into these three groups using \citet{angus2019}
gyrochronology models, according to the scheme shown in the legend.
Only stars cooler than 5000 K are included in the bottom panel in order to
isolate populations above and below the period gap, which only extends up to a
temperature of $\sim$4600 K.
In general, velocity dispersion increases with rotation period because both
quantities increase with age.
There is a smooth transition in velocity dispersion between stars with
rotation periods below and above the gap (orange squares to blue circles),
suggesting that these groups are part of the same galactic population.
Previously, only the overall velocity dispersions of all stars above and below
the gap have been compared, leading to the assumption that these groups belong
to two distinct populations \citep{mcquillan2014}.
The velocity dispersion of stars with rotation periods shorter than the lower
edge of the rotation period distribution (green triangles) is relatively
large.
This is an indication that many of these are synchronized binaries.
The rotation periods of stars in synchronized binaries are tidally locked to
the orbital period of the binary system and since synchronized binaries have
short orbital periods this can produce old, seemingly isolated stars, with
short rotation periods.
The large velocity dispersions of the most rapidly rotating stars indicates
that some fraction of these stars are old but rotating rapidly, and therefore
likely to be synchronized binaries.
Figure \ref{fig:gap} indicates that there is an increased probability of stars
with rotation periods less than $\sim$10 days being synchronized binaries.
This result is in agreement with a recent study which found that a large
fraction of visual binaries were rapid rotators, and the probability of a star
being a synchronized binary system substantially increased at rotation periods
of around 7 days \citep{simonian2019}.

% We calculated the overall velocity dispersions of stars as a function of
% rotation period and gyrochronal age.

% Each star is assigned the same velocity dispersion as in figure
% \ref{fig:vplot}, and because stars in the same temperature group are not
% necessarily in the same mass group, which is why stars of the same mass and period do not nece

% The shape of the period-\teff\ relations at old ages appears to follow the
% shape of the upper detection edge.
% The `M dwarf dip' \citep{vansaders2018}, a feature of the \mct\ rotation
% period catalog characterized by a dearth of slowly rotating M dwarfs between
% $\sim$3750 and 4250 K, is reflected in the lines of constant velocity
% dispersion (and presumed age) in the top panel of figure \ref{fig:vplot}.
% If the shape of the upper edge of rotation period measurements is created by a
% detection limit, it could indicate the rotation periods at which stars of
% different temperatures become relatively inactive (and their rotation periods
% therefore become undetectable).
% % Once inactive, stars would have little photometric variability induced by
% % star spots and their rotation periods would be difficult to measure, so may
% % not feature in the \mct\ catalog.
% If so, figure \ref{fig:dispersion_period_teff} suggests that stars cooler than
% $\sim$4500 K stay active longer than stars hotter than $\sim$4500 on average.

% The velocity dispersions of the coolest stars may be affected by a selection
% bias -- these extremely faint stars are more difficult to detect at larger
% distances, larger heights above the plane, and therefore larger velocities.
% It is possible that some high velocity stars of temperatures cooler than
% $\sim$4000 K are missing from this sample, and the velocity dispersions may
% therefore appear lower than they truly are.
% This could be why the velocity dispersion appears to decrease towards the
% right of figure \ref{fig:dispersion_period_teff}.

% \begin{figure}
%   \caption{
%     Similar to figure \ref{fig:vplot}, with mass instead of \teff\ on the
%     x-axis.
%     Masses are from the \kepler\ input catalog.
%     The white lines show the \citet{spada2019} rotational evolution models at
%     0.5, 1, 1.5, 2, 2.5, 4 and 4.57 Gyr, where age increases with rotation
%     period.
%     These models include age and mass-dependent coupling between the stellar
%     core and envelope.
%     The oldest models (4 and 4.57 Gyrs), at the largest rotation periods, show
%     an inversion at $\sim$0.7-0.5 M$_\odot$, where rotation period briefly
%     decreases with decreasing mass.
%     A similar phenomenon is visible in the velocity dispersions of field stars
%     shown as colored points in the background of this figure.
% }
%   \centering
%     \includegraphics[width=1\textwidth]{masses_with_models}
% \label{fig:masses_with_models}
% \end{figure}

% \subsection{The period gap}
% \label{sec:period_gap}

% The origin of the rotation period gap, first identified
% by \citet{mcquillan2013} and visible in figures \ref{fig:age_cut} and
% \ref{fig:dispersion_period_teff} still remains a mystery.
% This gap can be seen as an under-density of points between the 0.7-1.0 and
% 1.0-1.5 Gyr age ranges in figure \ref{fig:age_cut} and roughly follows a line
% of constant gyrochronal age of around 1.1 Gyr \citep[according to the
% gyrochronology relation of][]{angus2019}, as shown in figure
% \ref{fig:dispersion_period_teff}.
% Several explanations for the gap's origin have been proposed, including a
% discontinuous star formation history \citep{mcquillan2013, davenport2017,
% davenport2018}, a rapid change in magnetic field structure
% \citep{reinhold2019}, and erroneous rotation period measurements that are
% incorrect by a factor of two \citep{koen2018}.
% The latter explanation can be ruled out because stars below the gap have
% smaller velocity dispersions than the stars above the gap, indicating that
% they are kinematically younger \citep{mcquillan2013, davenport2018}, as
% evident in figures \ref{fig:age_cut} and \ref{fig:dispersion_period_teff}.
% For stars below the gap, in the 0.7-1.0 Gyr age range shown in figure
% \ref{fig:age_cut}, velocity dispersion is relatively constant as a function of
% temperature, however above the gap, in the 1.0-1.5 Gyr age range and older,
% velocity dispersion increases with \teff.
% The coolest stars in the 1.0-1.5 Gyr age range have the same velocity
% dispersion as the hottest stars in the age range above which indicates that
% the period-\teff\ relations are {\it flat} at these rotation periods.
% This is also visible in figure \ref{fig:dispersion_period_teff}.
% Below the gap, velocity dispersion within a given period range appears to {\it
% decrease} with decreasing temperature.
% The opposite appears to be true above the gap.
% It could be that the gap is positioned at a significant Rossby number/age at
% which stellar magnetic dynamos go through a transition.
% Perhaps before the age of 1.1 Gyr, or at Rossby numbers less than ???,
% magnetic braking is more efficient for stars with deeper convection zones.
% Once stars reach this critical age or Rossby number their magnetic fields
% undergo some radical transition, which produces the gap in the rotation
% period-\teff\ plane.
% After this transition, magnetic braking efficiency no longer increases with
% decreasing mass.
% Of course, it may be a coincidence that the gyrochronology relations seem to
% only flatten off above the period gap and we lack a sufficient quantity of
% data to do more than speculate here.
% New rotation periods from the \ktwo\ and \tess\ missions may be able to
% validate or rule out this hypothesis in the future.

% In the $\sim$ 1.1 Gyr NGC 6811 cluster, the rotation periods of mid-K dwarfs
% are faster than expected; their rotational evolution appears to have stalled,
% and they have similar rotation periods to the 650 Myr Praesepe cluster
% \citep{curtis2019}.
% The rotation periods of the K dwarfs in this cluster are plotted in figure
% \ref{fig:dispersion_period_teff}.
% Although NGC 6811's G dwarfs fall on the 1.1 Gyr gyrochronology model, the K
% dwarfs lie only a little above the 0.65 Gyr gyrochronology model.
% NGC 6811 straddles the rotation period gap: its G dwarfs lie above it and its
% K dwarfs lie below it.
% This cluster may be the `missing link' that connects two epoch of stellar
% spin-down.
% % , however without more observations of middle-aged open clusters it
% % : an early stage where the period-\teff\ relation for cool dwarfs has
% % a negative slope and a late stage where it has a positive slope.
% % In this case the period gap may delineate the transition between these two
% % regimes and is the point at which stellar magnetic dynamos likely undergo a
% \subsection{Velocity dispersion of coeval groups}
% \label{sec:age_cut}
% % dramatic structural shift at an age of $\sim$ 1.1 Gyr.
