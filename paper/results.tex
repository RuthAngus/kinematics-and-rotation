% word limit: 500
\begin{itemize}
\item{Showing that the gap stars are young.}
\item{Is there a lack of old M dwarfs is this sample?}
\end{itemize}

\begin{figure}
  \caption{
Rotation period vs. velocity in the galactic latitute direction.
}
  \centering
    \includegraphics[width=1\textwidth]{rotation_vb_dispersion}
\label{fig:rotation_vb_dispersion}
\end{figure}

\begin{figure}
  \caption{
Age vs. velocity in the galactic latitute direction.
The velocity of stars as a function of their age.
}
  \centering
    \includegraphics[width=1\textwidth]{age_vb_dispersion}
\label{fig:age_vb_dispersion}
\end{figure}
figure \ref{age_vb_dispersion} shows the velocity of stars in the \bvector\
direction, plotted against their gyrochronal ages.
These ages were calculated using equation 1 of \citet{stardate_paper},
implemented in the \sd\ \python\ package \citep{stardate}.
The two vertical lines show the approximate locations of the synchronized
binary upper limit and the rotation gap.
Stars to the left of the rotation gap (age $\lesssim$ 1 Gyr) and to the right
of the synchronized binary regime (age $\gtrsim$ 0.5 Gyr) have lower $V_b$
dispersion than stars to the right of the rotation gap (age $gtrsim$ 1 Gyr),
indicating that they are kinematically young.
This increase in velocity dispersion can be seen in both the individually
plotted stars, and in the bins of standard deviation shown as the solid black
line, which increase with age.
This figure clearly indicates that stars with rotation periods that indicate
they are young (but are greater than 7 days), are indeed young.
This rules out the possibility that the rotation period gap is caused by
either incorrect rotation periods, or binarity.

This figure also shows a lack of M dwarfs at old ages caused by the `M dwarf
dip' \citep{vansaders2019, mcquillan2014}.
This is a feature of the \citet{mcquillan2014} sample where there appears to
be a lack of slowly rotating stars at temperatures cooler than 4000 K.
Like the rotation period gap, the origin of the M dwarf dip could either be
due to a lack of stars with long rotation periods, or due to a suppression in
the amplitudes of variability for these stars.
However, in general the completeness of the \citet{mcquillan2014} catalog
increases as a function of decreasing temperature, and is around 80\% for M
dwarfs \citep{mcquillan2013, mcquillan2014, simonian2019}.
This suggests that the age-rotation relation has a different shape at
different ages, as indicated by the 1 Gyr NGC 6811 cluster that has a
different rotation period-age relation to those of younger clusters
\citep{curtis2019}.
