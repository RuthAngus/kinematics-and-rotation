The rotational evolution of cool dwarfs is poorly constrained after $\sim$1-2
Gyr due to a lack of precise ages and rotation periods for old main-sequence
stars.
In this work, we use the velocity dispersions of low-mass \kepler\ dwarfs as
an age proxy, to reveal their rotational evolution and demonstrate that
kinematics could be a useful tool for calibrating gyrochronology.
We find that a gyrochronology model, calibrated to fit the period--\teff\
relationship of the Praesepe cluster, does not apply to stars older than
around 1 Gyr.
Although late-K dwarfs spin more slowly than early-K and late-G dwarfs when
they are young, at old ages we find that late-G and early-K dwarfs rotate at
the {\it same rate} or faster than late-K dwarfs of the same age.
This result agrees qualitatively with semi-empirical models that vary the rate
of surface-to-core angular momentum transport as a function of time and mass.
It also aligns with recent observations of stars in the NGC 6811 cluster,
which indicate that K dwarfs experience an epoch of stalled spin-down.
We find that the oldest \kepler\ stars with measured rotation periods are
late-K and early-M dwarfs, indicating that these stars maintain spotted
surfaces and stay magnetically active longer than more massive stars.
Finally, based on their kinematics, we confirm that many rapidly rotating
K-dwarfs are likely to be synchronized binaries.
