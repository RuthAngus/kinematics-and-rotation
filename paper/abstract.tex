% Word limit: 250

The rotational evolution of cool dwarfs is mostly unconstrained after around
2-3 billion years, and the gyrochronology relations are poorly calibrated
beyond this time due to a lack of precise ages and rotation periods for old
main-sequence stars.
However, recent evidence suggests that the relationship between rotation
period and color/temperature/mass does not remain constant over large
timescales, contradicting a long-held assumption that underlies many
gyrochronology relations.
We use the velocities of populations of cool dwarfs to reveal the rotational
evolution of these stars at old ages.
We find that groups of stars, selected to be the same age using a
gyrochronology relation calibrated to the Praesepe cluster, do {\it not} have
the same velocity dispersion at all temperatures; cooler stars have greater
velocity dispersions than hotter stars.
We interpret this to mean that these cooler stars are older than the hotter
ones, which implies that the period-color relation of Praesepe is not
appropriate for old stars.
Futhermore, our results indicate that the slope of the period-color relation
actually {\it inverts}: at young ages early K stars rotate more rapidly than
late K stars, but beyond $\sim$ 2 Gyr they rotate more slowly.
These observations could be reproduced by mass-dependent dynamical heating,
where lower-mass stars experience a greater rate of dynamical heating than
higher-mass stars, however we find no evidence for mass-dependent heating
within K and M dwarfs in the \kepler\ field.
