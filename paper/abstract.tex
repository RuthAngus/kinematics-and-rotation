% Word limit: 250

Due to a lack of precise ages and rotation periods for old main-sequence
stars, the rotational evolution of FGKM dwarfs is mostly unconstrained
after around 2-3 billion years.
% This limit is set by the oldest open clusters whose members have precise light
% curves, and therefore photometric rotation periods.
Recent evidence suggests that the relationship between rotation period and
color/temperature/mass does not remain constant over large timescales,
contradicting the long-held assumption that underlies many gyrochronology
relations.
% For example, old clusters appear to have a flatter relationship between
% rotation period and color than young clusters.
In this work we use the kinematic ages of populations of K dwarfs to reveal
the evolving relationship between rotation period and effective temperature.
We find that groups of stars, selected to be the same age, do {\it not} have
the same velocity dispersion at all temperatures; cooler stars have greater
velocity dispersions than hotter stars.
Two different scenarios could be responsible for producing this observation:
either the rotation period-color relation changes over time, or stars of
different masses experience different rates of dynamical heating.
We show however, that if mass-dependent heating is responsible, the
differential heating rate is not constant over logarithmic time intervals
which implies that different heating processes must be acting before and after
ages of around 1 Gyr.
Perhaps a more likely scenario is that the relation between rotation period
and color changes over time, undergoing a particularly sharp transition at
around 1 Gyr.
If we assume that mass-dependent heating does not affect the data, our results
indicate that the slope of the period-color relation actually {\it inverts}:
at young ages early K stars rotate more rapidly than late K stars, but beyond
$\sim$ 2 Gyr they rotate more slowly.
% (similar to the age of the rotation gap and the 1.1 Gyr NGC 6811
% cluster, shown to have a different period-color relation to Praesepe).
% We find that the ages of field stars with measured rotation periods do not
% exceed 4-6 Gyrs, resolving a long-standing degeneracy between age and magnetic
% braking efficiency.
% We also show that stars rotating just below the rotation period gap are
% dynamically young, ruling out the possibility that the rotation period gap is
% caused by incorrect period measurements or binary companions.
% The main result of this paper could be produced by either an evolving
% period-color relationship {\it or} mass-dependent dynamical heating.
% Although we provide some evidence to suggest that the former effect is
% dominant, we cannot be sure and therefore caution that the results presented
% here are subject to this caveat.
