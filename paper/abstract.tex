% Word limit: 250

Due to a lack of precise ages for old stars with measured rotation periods,
the rotational evolution of FGKM dwarfs is still mostly unconstrained after
around 2-3 billion years.
% This limit is set by the oldest open clusters whose members have precise light
% curves, and therefore photometric rotation periods.
Recent evidence suggests that the relationship between rotation period and
color, temperature, and mass does not remain constant over large timescales,
contradicting the long-held assumption that underlies many gyrochronology
relations.
% For example, old clusters appear to have a flatter relationship between
% rotation period and color than young clusters.
In this work we use the kinematic ages of star populations to reveal the
evolving relationship between rotation period and effective temperature.
Not only does the period-color relation change over time, its slope appears to
{\it invert}: at young ages early K stars rotate more rapidly than late K
stars, but at old ages they rotate more rapidly.
% Although the gyrochronology relations fail to correctly predict the ages of
% old cool stars, we show that gyrochronology accurately predicts the ages of
% early K dwarfs, and young K dwarfs of all temperatures and those ages are
% consistent with ages predicted from age-velocity dispersion relations.
We find that the ages of field stars with measured rotation periods do not
exceed 4-6 Gyrs, resolving a long-standing degeneracy between age and magnetic
braking efficiency.
% It was previously unknown whether the upper-limit of the stellar rotation
% period distribution was created by a ceasura of magnetic braking (\ie old
% stars do not rotate more slowly after a critical age) or an inability to
% measure rotation periods for old stars.
% We show that the lack of long rotation periods is caused by enhanced
% measurement difficulty.
We also show that stars rotating just below the rotation period gap are
dynamically young, ruling out the possibility that the rotation period gap is
caused by incorrect period measurements or binary companions.
% Finally, we suggest that this gap may mark a critical point in the rotation
% period evolution of K dwarfs.
The main result of this paper could be produced by either an evolving
period-color relationship {\it or} mass-dependent dynamical heating.
Although we provide some evidence to suggest that the former effect is
dominant, we cannot be sure and therefore caution that the results presented
here are subject to this caveat.
