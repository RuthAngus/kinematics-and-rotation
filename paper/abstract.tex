The rotational evolution of FGKM dwarfs is still mostly unconstrained after
around 2-3 billion years.
This limit is set by the oldest open clusters whose members have precise light
curves, and therefore photometric rotation periods.
However, recent evidence suggests that the relationship between rotation
period and color/temperature/mass does not remain constant over large
timescales, as has long been assumed.
Old clusters seem to have a flatter relationship between rotation period and
color than young clusters.
In addition, old field stars with asteroseismic ages, observed by the \kepler\
spacecraft, appear to be rotating more rapidly than expected for their ages
and masses.
In this work we use the velocity dispersion of groups of stars as a proxy for
age in order to demonstrate that the relationship between rotation period and
effective temperature not only changes over time but even appears to invert.
At young ages G stars rotate more rapidly than K and M dwarfs, but at old ages
they appear to rotate more rapidly.
In addition, we show that stars rotating just below the rotation period gap
are dynamically young, we ruling out the possibility that the rotation period
gap is caused by incorrect period measurements or binary companions.
