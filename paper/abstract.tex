% The distribution of rotation periods of K and M stars, measured from light
% curves obtained from the \kepler\ spacecraft, has a sharp mass-dependent gap
% at around 10-20 days.
% This gap traces a line of constant age and constant Rossby number in the
% rotation period-effective temperature plane, indicating that the cause could
% be related to a discontinuity in either the local star formation history, or
% the magnetic braking evolution of stars.
% A third explanation for the rotation period gap is measurement error caused by
% confounders such as binary companions or aliasing.
% For example, the lower rotation sequence could be a reflection of the upper
% sequence, caused by incorrect measurements at half the true period.
% In this paper, we rule out the possibility that this gap could be caused by
% incorrect period measurements or binary companions, by showing that the
% rapidly rotating stars are dynamically young.

% We also find that nai/"vely applying gyrochronology relations to stars hotter
% than around 5000 K can result in inaccurate ages.

% Finally, we show that the changing shape of the rotation-temperature relations
% causes the `M dwarf dip', providing further evidence that the gyrochronology
% relations are not separable functions of color and age.

The rotational evolution of FGKM dwarfs is still mostly unconstrained after
around 2-3 billion years.
This limit is set by the oldest open clusters whose members have precise light
curves, and therefore photometric rotation periods.
However, recent evidence suggests that the relationship between rotation
period and color/temperature/mass does not remain constant over large
timescales, as has long been assumed.
Old clusters seem to have a flatter relationship between rotation period and
color than young clusters.
In addition, old field stars with asteroseismic ages, observed by the \kepler\
spacecraft, appear to be rotating more rapidly than expected for their ages
and masses.
In this work we use the velocity dispersion of groups of stars as a proxy for
age in order to demonstrate that the relationship between rotation period and
effective temperature not only changes over time but even appears to invert.
At young ages G stars rotate more rapidly than K and M dwarfs, but at old ages
they appear to rotate more rapidly.
In addition, we show that stars rotating just below the rotation period gap
are dynamically young, we ruling out the possibility that the rotation period
gap is caused by incorrect period measurements or binary companions.
