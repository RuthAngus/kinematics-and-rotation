% Word limit: 250

The rotational evolution of cool dwarfs is poorly constrained after $\sim$2-3
billion years due to a lack of precise ages and rotation periods for old
main-sequence stars.
% However, recent evidence suggests that the relationship between rotation
% period and color/temperature/mass does not remain constant over large
% timescales, contradicting a long-held assumption that underlies many
% gyrochronology relations.
In this work, we use the velocities of low-mass \kepler\ dwarfs to reveal
their rotational evolution and demonstrate that kinematics have potential to
be a useful tool for calibrating gyrochronology.
At ages less than $\sim$1 Gyr we find that a gyrochronology model, calibrated
to the Praesepe cluster, accurately predicts the relative ages of stars.
At these young ages, lower-mass stars spin more slowly than higher-mass stars
because their stronger magnetic fields lead to more efficient angular momentum
loss.
However, at old ages we find that late G and early K dwarfs rotate {\it
faster} than late K dwarfs of the same age.
% These oThis may be a feature of angular momentum redistribution in the stellar
% interior.
% This implies that low-mass stars lose angular momentum more slowly than
% higher-mass stars after a certain age or Rossby number.
These results, based on field star rotation periods, align with recent
findings from the rotation periods of stars in middle-aged open clusters and
theoretical models that vary the rate of surface-to-core angular momentum
transport as a function of time and mass.
Finally, we find no evidence for mass-dependent heating in a sample of K and M
dwarfs in the \kepler\ field.

% We find that groups of stars, selected to be the same age using a
% gyrochronology relation calibrated to the Praesepe cluster, do {\it not} have
% the same velocity dispersion at all temperatures; cooler stars have greater
% velocity dispersions than hotter stars.
% We interpret this to mean that these cooler stars are older than the hotter
% ones, which implies that the period-color relation of Praesepe is not
% appropriate for old stars.
% Futhermore, our results indicate that the slope of the period-color relation
% actually {\it inverts}: at young ages early K stars rotate more rapidly than
% late K stars, but beyond $\sim$ 2 Gyr they rotate more slowly.
% These observations could be reproduced by mass-dependent dynamical heating,
% where lower-mass stars experience a greater rate of dynamical heating than
% higher-mass stars, however we find no evidence for mass-dependent heating
% within K and M dwarfs in the \kepler\ field.
