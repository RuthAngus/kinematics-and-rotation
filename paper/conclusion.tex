\section{Conclusion}

We examined the gyrochronology relations using the velocity dispersions of
field stars with measured rotation periods.
We found that old groups of K dwarfs selected to be coeval using the
\citet{angus2019} gyrochronology relation do {\it not} have the same velocity
dispersion across all temperatures.
It appears that the period-\teff\ relation changes shape over time: rotation
period is negatively correlated with \teff\ at young ages, and positively
correlated at old ages.
The relation appears to start flattening out after $\sim$1 Gyr (see figure
\ref{fig:age_cut}), potentially around the same age, or just older than the
period gap which is located at a (gyro) age of around $\sim$1.1 Gyr.
This gap may represent a significant transitional epoch in the magnetic
behavior of stars.
Finally, when velocity dispersion is interpreted as an age proxy, it appears
that the oldest stars in the \mct\ catalog are cooler than 4500 K, which
suggests that lower-mass stars remain active for longer and their rotation
periods can be measured at older ages.
% This indicates that either the \citep{angus2019} period-color relation is
% incorrect, or that mass-dependent dynamical heating is affecting this data set.
% We presented three pieces of evidence that suggest mass-dependent heating
% cannot be responsible for all of the increasing velocity dispersion.

% \begin{itemize}

% \item{
%         No strong evidence for mass-dependent heating has been found among M, L
% and T-type stars.
% These low-mass stars live longer than the current age of the Universe and
%         therefore do not suffer from a mass-age degeneracy
%         \citep{faherty2009}.
% }

% \item{
%         Literature measurements of the (\vz) AVR exponent are slightly higher
% than the value measured from our data set, but calculated using more massive
% stars.
% Mass-dependent dynamical heating should however produce a {\it smaller} AVR
%         exponent for higher-mass stars.
% We acknowledge that the stars in the \mct\ sample, analyzed in this work, are
%         subject to different selection biases than those used to calculate
%         AVRs in the literature \citep{holmberg2009, aumer2009, yu2018}, so
%         drawing direct comparisons between them does not provide a reliable
%         test.
% }

% \item{
%         The differences in the (\vb) AVR exponent between the hottest and
% coolest K dwarfs in our sample are extremely large, spanning values between
% 0.4 and 0.75, yet the difference in mass is less than a factor of two.
% Such a large spread in AVR exponents cannot be explained by mass-dependent
%         heating alone.
% }

% \end{itemize}

% \racomment{Words about the significance of an inverted period-color relation.}

% \racomment{How does the weakened van Saders braking law fit into all this?}

% \racomment{Talk about the fact that the stars in each bin are not necessarily
% the same age.}

% \racomment{Dispersion could be dependent on the selection function.
% We cut stars at high b, because we find that these do lead to a bias.
% We find no evidence for increased velocity dispersion with decreasing teff in
% either vb or vz.
% We find that sigma clipping is necessary for vb but not vz and do not find
% that the outliers are distributed towards higher galactic latitudes.
% }

% \racomment{Trevor suggest that stars forming in clusters could be the reason
% why young stars are so dynamically cool.}

% \racomment{Are there any asteroseismic stars in this sample that could help
% pin down the age?}
