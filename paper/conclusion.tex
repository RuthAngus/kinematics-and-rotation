\section{Conclusion}

We found that groups of K dwarfs, selected to be coeval using the
\citet{angus2019} gyrochronology relation, do {\it not} have the same velocity
dispersion across all temperatures.
This indicates that {\it either} the \citep{angus2019} period-color relation
is wrong, {\it or} that mass-dependent dynamical heating is affecting this
data set.
We presented three pieces of evidence that suggest mass-dependent heating
cannot be responsible for all of the increasing velocity dispersion.

\begin{itemize}

\item{No strong evidence for mass-dependent heating has been found among M, L
and T-type stars.
These low-mass stars live longer than the current age of the Universe and
therefore do not suffer from a mass-age degeneracy \citep{faherty2009}.}

\item{Literature measurements of the (\vz) AVR exponent are slightly higher
than the value measured from our data set, but calculated using more massive
stars.
Mass-dependent dynamical heating should however produce a {\it smaller} AVR
exponent for higher-mass stars.
We acknowledge that the stars in the \mct\ sample, analyzed in this work, are
subject to different selection biases than those used to calculate AVRs in the
literature \citep{holmberg2009, aumer2009, yu2018}, so drawing direct
comparisons between them is not a robust test.}

\item{The differences in the (\vb) AVR exponent between the hottest and
coolest K dwarfs in our sample are extremely large, spanning values between
0.3 and 0.65, yet the difference in mass is less than a factor of two.
Such a large spread in AVR exponents cannot be explained by mass-dependent
heating alone.}

\end{itemize}

\racomment{Words about the significance of an inverted period-color relation.}
