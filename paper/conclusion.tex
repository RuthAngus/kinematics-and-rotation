\section{Conclusion}

We examined the gyrochronology relations using the velocity dispersions of
field stars with measured rotation periods.
We found that old groups of cool dwarfs selected to be coeval using the
\citet{angus2019} gyrochronology relation do {\it not} have the same velocity
dispersion across all temperatures.
This implies that the \citet{angus2019} relation, which is based on the
period-color relation of the 650 Myr Praesepe cluster, does not correctly
describe the period-age-color/\teff\ relation for old stars.
It appears that the period-color/\teff\ relation changes shape over time, \ie\
the period-color/\teff\ relation and the period-age relation are
non-separable.
At ages younger than $\sim$1 Gyr, rotation period is anti-correlated with
\teff: cooler stars spin more slowly than hotter stars of the same age.
However, at old ages the opposite is true: cooler stars spin more rapidly than
hotter stars of the same age.
The period-color/\teff\ relation appears to start flattening out after $\sim$1
Gyr (see figure \ref{fig:age_cut}), potentially around the same age, or just
older than the period gap which is located at a gyrochronal age of around
$\sim$1.1 Gyr (see figure \ref{fig:dispersion_period_teff}).
This gap may represent a significant transitional epoch in the magnetic
behavior of stars.
Finally, when velocity dispersion is interpreted as an age proxy, it appears
that the oldest stars in the \mct\ catalog are cooler than 4500 K, which
suggests that lower-mass stars remain active for longer and their rotation
periods can be measured at older ages.
We outlined a number of scenarios which could provide alternative
explainations for these observations, including incorrect dust corrections for
the lowest-mass stars and an excess of companions increasing the velocity
dispersion for these stars.

If the period-color/\teff\ relation does invert at old ages as our results
suggest, this would be a paradigm shift for gyrochronology.
Stellar spin-down rate is thought to be directly tied to magnetic field
strength, and the deeper convection zones of cooler stars generate stronger
magnetic fields which {\it should} lead to more efficient angular momentum
loss.
However, the micro- and macro-physics of stellar structure and evolution and
magnetic dynamo models are extremely complicated and a lot is still unknown
about the magnetic behavior of stars.
Observations like these can provide useful constraints for physical models,
and may help to reveal new physical processes at work in stars like our own
Sun and other planet hosts.
