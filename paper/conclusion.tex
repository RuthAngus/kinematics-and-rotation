\section{Conclusion}

In this paper, we demonstrated that the dispersion of velocities in the
direction of galactic latitude, \vb, can be used as an age proxy, by showing
that there is no strong evidence for mass-dependent heating in low-mass
\kepler\ dwarfs: the velocity dispersions of K and M dwarfs, whose
main-sequence lifetimes are longer than around 11 Gyrs, do not appear to
increase with decreasing mass.
Although {\it vertical} velocity, \vz, is a quantity that has been
demonstrated to trace time-dependent orbital heating in the disc of the
Galaxy, most stars with measured rotation periods do not yet have radial
velocities, so we used velocity in the direction of Galactic latitude, \vb\,
as a proxy for \vz.
Using stars in the GUMS simulation, we showed that using \vb\ as a proxy for
\vz\ introduces an additional velocity dispersion, which increases with
increasing Galactic latitude.
For this reason we did not attempt to convert \vb\ dispersions into ages using
an age-velocity dispersion relation.
However, after removing high-latitude ($b>15^\circ$) stars from the sample, we
confirmed that using \vb\ instead of \vz\ does not introduce any
mass-dependent velocity dispersion bias into the sample.
We therefore assumed that \vb\ velocity dispersion can be used to accurately
rank stars by age, \ie\ a group of stars with a large velocity dispersion is,
on average, older than a group of stars with a small velocity dispersion.

% We found that old groups of cool dwarfs, selected to be coeval using the
% \citet{angus2019} gyrochronology relation, do {\it not} have the same velocity
% dispersion across all temperatures.
We used the \vb\ velocity dispersions of stars in the \mct\ catalog to explore
the evolution of stellar rotation period as a function of effective
temperature and age.
We found that the \citet{angus2019} relation, which is based on the
period-color relation of the 650 Myr Praesepe cluster, does not correctly
describe the period-age-\teff\ relation for old stars.
Instead we found that, at young ages, rotation period is anti-correlated with
\teff: cooler stars spin more slowly than hotter stars of the same age.
However, at intermediate ages the relation flattens out and K dwarfs rotate at
the same rate, regardless of mass.
At old ages, it seems that cooler K dwarfs spin more rapidly than hotter K
dwarfs of the same age.
We showed that the period-\teff\ relations change shape over time in a way
that qualitatively agrees with theoretical models which include a
mass-dependent core-envelope angular momentum transport \citep{spada2019}.

% The period-color/\teff\ relation seems to start flattening out after $\sim$1
% Gyr (see figure \ref{fig:age_cut}), potentially around the same age, or just
% older than the period gap which is located at a gyrochronal age of around
% $\sim$1.1 Gyr (see figure \ref{fig:dispersion_period_teff}).
We also found that the oldest stars in the \mct\ catalog are cooler than 4500
K, which suggests that lower-mass stars remain active for longer, allowing
their rotation periods to be measured at older ages.
We speculate that the rotation period gap \citep{mcquillan2014} may separate
a young regime where stellar rotation periods decrease with increasing mass
from an old regime where periods increase with increasing mass, however more
data are needed to provide a conclusive result.
The velocity dispersions of stars increase smoothly across the rotation period
gap, indicating that the gap does not separate two distinct populations.
Finally, we used kinematics to indicate that there is a population of
synchronized binaries with rotation periods less than around 10 days.

% We outlined a number of scenarios which could provide alternative
% explanations for these observations, including incorrect dust corrections for
% the lowest-mass stars and an excess of companions increasing the velocity
% dispersion for these stars.

% If the period-color/\teff\ relation does invert at old ages as our results
% suggest, this would be a paradigm shift for gyrochronology.
% Stellar spin-down rate is thought to be directly tied to magnetic field
% strength, and the deeper convection zones of cooler stars generate stronger
% magnetic fields which {\it should} lead to more efficient angular momentum
% loss.
% However, the micro- and macro-physics of stellar structure and evolution and
% magnetic dynamo models are extremely complicated and a lot is still unknown
% about the magnetic behavior of stars.
% Observations like these can provide useful constraints for physical models,
% and may help to reveal new physical processes at work in stars like our own
% Sun and other planet hosts.
