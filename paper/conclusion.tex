\section{Conclusion}

We examined the gyrochronology relations using the velocity dispersions of
field stars with measured rotation periods.
We found that old groups of K dwarfs selected to be coeval using the
\citet{angus2019} gyrochronology relation do {\it not} have the same velocity
dispersion across all temperatures.
It appears that the period-\teff\ relation changes shape over time: rotation
period is negatively correlated with \teff\ at young ages, and positively
correlated at old ages.
The relation appears to start flattening out after $\sim$1 Gyr (see figure
\ref{fig:age_cut}), potentially around the same age, or just older than the
period gap which is located at a (gyro) age of around $\sim$1.1 Gyr.
This gap may represent a significant transitional epoch in the magnetic
behavior of stars.
Finally, when velocity dispersion is interpreted as an age proxy, it appears
that the oldest stars in the \mct\ catalog are cooler than 4500 K, which
suggests that lower-mass stars remain active for longer and their rotation
periods can be measured at older ages.
