\section{Conclusion}

We examined the rotational evolution of \kepler\ field stars by interpreting
their velocity dispersions as a proxy for age.
% We found that old groups of cool dwarfs, selected to be coeval using the
% \citet{angus2019} gyrochronology relation, do {\it not} have the same velocity
% dispersion across all temperatures.
We found that the \citet{angus2019} relation, which is based on the
period-color relation of the 650 Myr Praesepe cluster, does not correctly
describe the period-age-color/\teff\ relation for old stars.
It appears that the period-color/\teff\ relation changes shape over time in a
way that qualitatively agrees with theoretical models which include a
mass-dependent core-envelope angular momentum transport \citep{spada2019}.
At young ages, rotation period is anti-correlated with \teff: cooler stars
spin more slowly than hotter stars of the same age.
However, at intermediate ages the relation flattens out and K dwarfs rotate at
the same rate, regardless of mass.
At old ages, it seems that cooler K dwarfs spin more rapidly than hotter K
dwarfs of the same age.
% We speculate that the rotation period gap \citep{mcquillan2014} may separate
% a young regime where stellar rotation periods decrease with increasing mass
% from an old regime where periods increase with increasing mass, however more
% data are needed to provide a conclusive result.
% The period-color/\teff\ relation seems to start flattening out after $\sim$1
% Gyr (see figure \ref{fig:age_cut}), potentially around the same age, or just
% older than the period gap which is located at a gyrochronal age of around
% $\sim$1.1 Gyr (see figure \ref{fig:dispersion_period_teff}).
Finally, when velocity dispersion is interpreted as an age proxy, it appears
that the oldest stars in the \mct\ catalog are cooler than 4500 K, which
suggests that lower-mass stars remain active for longer, allowing their
rotation periods to be measured at older ages.
% We outlined a number of scenarios which could provide alternative
% explanations for these observations, including incorrect dust corrections for
% the lowest-mass stars and an excess of companions increasing the velocity
% dispersion for these stars.

% If the period-color/\teff\ relation does invert at old ages as our results
% suggest, this would be a paradigm shift for gyrochronology.
% Stellar spin-down rate is thought to be directly tied to magnetic field
% strength, and the deeper convection zones of cooler stars generate stronger
% magnetic fields which {\it should} lead to more efficient angular momentum
% loss.
% However, the micro- and macro-physics of stellar structure and evolution and
% magnetic dynamo models are extremely complicated and a lot is still unknown
% about the magnetic behavior of stars.
% Observations like these can provide useful constraints for physical models,
% and may help to reveal new physical processes at work in stars like our own
% Sun and other planet hosts.
