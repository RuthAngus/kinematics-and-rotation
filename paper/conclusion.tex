\section{Conclusion}

We examined the rotation period-color component of the \citep{angus2019}
gyrochronology relation and found that groups of stars, selected to be the
same age using this relation, do {\it not} have the same velocity dispersion
across all temperatures.
This indicates that {\it either} the \citep{angus2019} period-color relation
is wrong, {\it or} that mass-dependent dynamical heating is affecting this
data set.
We found that the exponent of the AVR relation, calibrated using F and G stars
in the Geneva-Copenhagen Survey, with ages estimated via isochrone-fitting, is
greater than the exponent we find for the AVR relation fit to the most massive
K dwarfs in our sample.
Since mass-dependent dynamical heating should lead to a larger AVR exponent
for lower-mass stars, this suggests that mass-dependent heating is not
strongly affecting our sample.
However, we used velocity in galactic latitude, \vb, not \vz, to calculate our
AVRs, and while \vb\ is a close approximation to \vz\ at low galactic
latitudes, it is not identical.
Indeed, the \vz\ AVR should be larger than the \vb\ AVR, since orbital
scattering is expected to have a strongest effect in the vertical direction.
For the 290 stars in our sample with \gaia\ radial velocity measurements we
measured the \vz\ AVR exponent to be larger than the \vb\ AVR exponent.

For now, we cannot draw a strong conclusion regarding the origin of the
observed increased velocity dispersion for stars of lower masses since either
an incorrectly calibrated gyrochronology relation, or mass-dependent dynamical
heating could be the cause.
