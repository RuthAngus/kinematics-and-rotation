% zip ms_files.zip ms.tex abstract.tex intro.tex method.tex results.tex conclusion.tex Praesepe.pdf variance.pdf simulated_CMD.pdf rotation_model_praesepe.pdf simulation_results.pdf NGC6819.pdf NGC6819_results.pdf hz.bib

\documentclass[useAMS, usenatbib, preprint, 12pt]{aastex}
% \documentclass[a4paper,fleqn,usenatbib,useAMS]{mnras}
\usepackage{cite, natbib}
\usepackage{float}
\usepackage{amsmath}
\usepackage{hyperref}
\usepackage{epsfig}
\usepackage{cases}
\usepackage[section]{placeins}
\usepackage{graphicx, subfigure}
\usepackage{color}
\usepackage{bm}

\AtBeginDocument{\let\textlabel\label}

\newcommand{\ie}{{\it i.e.}}
\newcommand{\eg}{{\it e.g.}}
\newcommand{\etal}{{\it et al.}}

\newcommand{\kepler}{{\it Kepler}}
\newcommand{\Kepler}{{\it Kepler}}
\newcommand{\corot}{{\it CoRoT}}
\newcommand{\Ktwo}{{\it K2}}
\newcommand{\ktwo}{\Ktwo}
\newcommand{\TESS}{{\it TESS}}
\newcommand{\tess}{{\it TESS}}
\newcommand{\LSST}{{\it LSST}}
\newcommand{\lsst}{{\it LSST}}
\newcommand{\Wfirst}{{\it WFIRST}}
\newcommand{\wfirst}{{\it WFIRST}}
\newcommand{\SDSS}{{\it SDSS}}
\newcommand{\PLATO}{{\it PLATO}}
\newcommand{\plato}{{\it PLATO}}
\newcommand{\Gaia}{{\it Gaia}}
\newcommand{\gaia}{{\it Gaia}}
\newcommand{\panstarrs}{{\it PanSTARRS}}
\newcommand{\LAMOST}{{\it LAMOST}}

\newcommand{\Teff}{$T_{\mathrm{eff}}$}
\newcommand{\teff}{$T_{\mathrm{eff}}$}
\newcommand{\FeH}{[Fe/H]}
\newcommand{\feh}{[Fe/H]}
\newcommand{\prot}{$P_{\mathrm{rot}}$}
\newcommand{\pmega}{$\bar{\omega}$}
\newcommand{\mj}{$m_j$}
\newcommand{\mh}{$m_h$}
\newcommand{\mk}{$m_k$}
\newcommand{\mx}{$m_x$}
\newcommand{\logg}{log(g)}
\newcommand{\dnu}{$\Delta \nu$}
\newcommand{\numax}{$\nu_{\mathrm{max}}$}
\newcommand{\degrees}{$^\circ$}
\newcommand{\vz}{$v_{\bf z}$}
\newcommand{\vb}{$v_{\bf b}$}
\newcommand{\kms}{$kms^{-1}$}
\newcommand{\sigmavb}{$\sigma_{v{\bf b}}$}
\newcommand{\sigmavz}{$\sigma_{v{\bf z}}$}

% \newcommand{\nsim_stars}{841}

\newcommand{\amnh}{1}
\newcommand{\cca}{2}
\newcommand{\hawaii}{3}

\newcommand{\sd}{{\tt stardate}}
\newcommand{\gcolor}{$G_{BP} - G_{RP}$}
\newcommand{\mcp}{\citep{mcquillan2014}}
\newcommand{\mct}{\citet{mcquillan2014}}
\newcommand{\bvector}{${\bf b}$}
\newcommand{\python}{{\it Python}}

\newcommand{\racomment}[1]{{\color{blue}#1}}

\begin{document}

\title{Exploring the ages of rotating stars using galactic dynamics: a novel
approach to calibrating gyrochronology}

\author{%
    Ruth Angus,
    Angus Beane,
    Adrian Price-Whelan,
    Jennifer van Saders,
    Elisabeth Newton,
    Jason Curtis,
    Travis Berger,
    Dan Foreman-Mackey,
    Lauren Anderson,
    Megan Bedell,
    Jackie Faherty,
    Rocio Kiman,
    Melissa Ness}

% \altaffiltext{\amnh}{American Museum of Natural History, Central Park West,
% Manhattan, NY, USA}
% \altaffiltext{\cca}{Center for Computational Astrophysics, Flatiron Institute,
% 162 5th Avenue, Manhattan, NY, USA}
% \altaffiltext{\hawaii}{Institute for Astronomy, University of Hawai'i at
% M\={a}noa, Honolulu, HI, USA}


% abstract.
\begin{abstract}

The distribution of rotation periods of K and M stars, measured from light
curves obtained from the \kepler\ spacecraft, has a sharp mass-dependent gap
at around 10-20 days.
This gap traces a line of constant age and constant Rossby number in the
rotation period-effective temperature plane, indicating that the cause could
be related to a discontinuity in either the local star formation history, or
the magnetic braking evolution of stars.
A third explanation for the rotation period gap is measurement error caused by
confounders such as binary companions or aliasing.
For example, the lower rotation sequence could be a reflection of the upper
sequence, caused by incorrect measurements at half the true period.
In this paper, we rule out the possibility that this gap could be caused by
incorrect period measurements or binary companions, by showing that the
rapidly rotating stars are dynamically young.
\end{abstract}

\section{Introduction}
\subsection{The rotation periods of field stars}
The rotation periods of middle-aged FGKM stars are mostly determined by their
mass, age and evolutionary stage, and this is due, for the most part, to
angular momentum loss through magnetic braking.
Although stars are born with a range of rotation periods, a steep dependence
of angular momentum loss, $J$ angular frequency, $\omega$, $J \propto
\omega^3$ causes stellar rotation periods to converge after a characteristic
timescale.
This timescale is on the order of a few tens of millions of years for F and G
stars, but hundreds of millions of years for K and M stars.
Once stars exceed this age, they steadily lose angular momentum via magnetized
stellar winds that remove angular momentum from the star at a radius that is
determined by the magnetic field strength.
The Rossby number, the ratio of rotation period to convective turnover
timescale, is linked to the strength of stellar magnetic dynamos.
Large-scale polodial and toroidal stellar magnetic fields are thought to be
generated by the $\alpha-\Omega$ mechanism, whereby convective motion of
plasma and differential stellar rotation cause the winding up of magnetic
field lines, generating a strong magnetic field \racomment{(citations)}.
In general, the ratio between a star's rotation period and its characteristic
timescale of convective motion (the Rossby number) determines the strength of
the magnetic field \racomment{(citation)}.
Stars that rotate quickly and have deep convection zones, therefore long
convective turnover times, have small Rossby numbers and strong magnetic
fields.
Young M dwarfs typically have the strongest magnetic fields.
Slowly rotating stars with shallow convective zones, such as old F and G
dwarfs, have weak magnetic fields \racomment{(citations)}.
Since magnetic field strength depends on stellar rotation period and mass via
the Rossby number, it follows that angular momentum loss driven by magnetized
stellar winds also depends on Rossby number.

The first large-scale stellar rotation period catalog, generated from
analyzing light curves from the \kepler\ spacecraft is the
\citet{mcquillan2014} catalog.
It contains around 34,000 rotation periods of FGKM dwarfs and subgiants,
measured by calculating autocorrelation functions (ACFs) of light curves.
This catalog revealed a gap in the rotation periods of K and M dwarfs: a strip
in the rotation vs effective temperature plane that is under-populated
compared to the surrounding parts of parameter space.
This gap can be seen in figure \ref{fig:the_gap}.
\begin{figure}
  \caption{
The rotation periods of 18,259 FGKM dwarf stars, measured by \mct, vs. their
effective temperatures.
We applied cuts to the \gaia\ color-magnitude diagram of these stars (figure
\ref{fig:CMD_cuts}) in order to remove equal-mass binaries and subgiants from
the sample.
The rotation period gap can be seen as an almost-horizontal gap toward the
right of this figure.
}
  \centering
    \includegraphics[width=1\textwidth]{the_gap}
\label{fig:the_gap}
\end{figure}
The rotation period gap is most prevalent for low-mass stars, of spectral type
K and M, but was recently shown to extend through G, F and A types as well
\citep{davenport2017}.

Three explanations for the origin of this have been proposed.
Firstly, since the gap falls along a rotational/gyrochronal isochrone, it
follows that it may be caused by a discontinuous age distribution within the
sample.
The gap could be created if there exist two populations of stars with
different age distributions, with the distribution of young ages peaking at
around 800 Myr, and the distribution of old ages peaking at around 2-3 billion
years.
This idea was investigated by \mct, who suggested that the two populations
might be the thin and thick disk of the Milky Way.
However, they found no sharp change in kinematic properties between the two
samples.
In addition, the age distributions of the thin and think disks are expected to
be around ... and ... years respectively: older than the majority of stars in
the \mct\ sample \racomment{(citations)}.

The second explanation for the rotation period gap is that stars do not lose
angular momentum uniformly over time: instead they undergo a period of rapid
spin down at an age of around 1.1 Gyr, or a Rossby number of around...
\racomment{(Calculate this!)}.
Since the rotation period gap is located along a line of constant Rossby
number \racomment{(what is this number?)}, and since the rate of stellar
angular momentum loss is known to be related to magnetic field strength (this
is the principle behind gyrochronology \citep[\eg][]{kawaler1989,
pinsonneault1989, barnes2003, barnes2007, angus2015, vansaders2016,
vansaders2018, angus2019}), it follows that the gap could be caused
by some phenomenon related to magnetic braking.

As of May 2019, the \citet{mcquillan2014} catalog has been used more than 250
times, for a range of studies spanning fields from exoplanets to white dwarfs,
to galactic archaeology.
Since it underpins so many astronomical studies, a large number of incorrectly
measured rotation periods in this catalog would undermine several of the
studies built upon it.

\subsection{Using kinematics to solve the mystery}

Why does proper motion in galactic latitude work for this problem?

\bibliographystyle{plainnat}
\bibliography{bib.bib}
\end{document}
