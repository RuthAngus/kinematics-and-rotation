% 270 words
% Limit: 500 words

\section{Method}
\label{sec:method}

\subsection{The data}
\label{sec:the_data}

We used the publicly available \kepler-\gaia\ DR2 crossmatched
catalog\footnote{Available at gaia-kepler.fun} to combine the \mct\ catalog of
stellar rotation periods, measured from \kepler\ light curves, with the \gaia\
DR2 catalog of parallaxes, proper motions and apparent magnitudes.
Reddening and extinction from dust was calculated for each star using the
Bayestar dust map implemented in the {\tt dustmaps} {\it Python} package
\citep{green2018}, and {\tt astropy} \citep{astropy2013}.

For this work, we used the precise \textit{Gaia} DR2 photometric color,
$G_{\rm BP} - G_{\rm RP}$, to estimate \teff\ for the Kepler rotators.
Curtis \etal\ (2020, in prep) combined effective temperature measurements for
nearby, unreddened field stars in benchmark samples, including FGK stars
characterized with high-resolution optical spectroscopy \citep{brewer2016}, M
dwarfs characterized with low-resolution optical and near-infrared
spectroscopy \citep{mann2015}, and K and M dwarfs characterized with
interferometry and bolometric flux analyses \citep{boyajian2012}.
This empirical color--temperature relation is valid over the color range $0.55
< (G_{\rm BP} - G_{\rm RP})_0 < 3.20$, corresponding to $6470 < T_{\rm eff} <
3070$~K.
The dispersion about the relation implies a high precision of 50~K.
These benchmark data enable us to accurately estimate \teff\ for cool dwarfs
\citep[\eg][]{rabus2019}, and allows us to correct for interstellar reddening
at all temperatures\footnote{The color--temperature relation is described in
detail in the Appendix of, and the formula is provided in Table 4 of, Curtis
\etal\ (2020, in prep).}.
The equation we used to calculate photometric temperatures from Gaia \gcolor\
color is a seventh-order polynomial with coefficients given in table
\ref{tab:coeffs}.
% \ref{tab:coeffs}.
% \begin{equation}
%     \mathrm{T_{eff}} = a + bC cC^2 + dC^3
%     eC^4 + 15598.5C^5 -2694.76C^6 + 192.865C^7,
% % Jason's updated parameters:
% % −416.585
% % 39780.0
% % −84190.5
% % 85203.9
% % −48225.9
% % 15598.5
% % −2694.76
% % 192.865
%     % \mathrm{T_{eff}} = 8960 -4802C + 1931C^2 -2446C^3 + 2669C^4 - 1324C^5 +
%     % 301C^6 - 26C^7,
% % 8959.8112335205078, -4801.5566310882568, 1931.4756631851196,
% %           -2445.9980716705322, 2669.0248055458069, -1324.0671020746231,
% %           301.13205924630165, -25.923997443169355]
% \label{eq:curtis}
% \end{equation}
% where C is \gaia\ \gcolor and the coefficients are provided in table
% \ref{coeffs}.
\begin{table}[h!]
  \begin{center}
      \caption{
          Coefficient values for the 7th-order polynomial used to estimate
      \teff\ from \Gaia\ \gcolor\ color, calibrated in Curtis \etal\ (2020, in
      prep).}
    \label{tab:coeffs}
    \begin{tabular}{l|c} % <-- Alignments: 1st column left and 2nd middle, with vertical lines in between
        (\gcolor\ ) exponent & Coefficient  \\
      \hline
      $0$ & -416.585 \\
      $1$ & 39780.0  \\
      $2$ & -84190.5 \\
      $3$ & 85203.9  \\
      $4$ & -48225.9 \\
      $5$ & 15598.5  \\
      $6$ & -2694.76 \\
      $7$ & 192.865  \\
    \end{tabular}
  \end{center}
\end{table}

Photometric binaries and subgiants were removed from the \mct\ sample by
applying cuts to the color-magnitude diagram (CMD), shown in figure
\ref{fig:age_gradient}.
A 6th-order polynomial was fit to the main sequence and raised by 0.27 dex to
approximate the division between single stars and photometric binaries (shown
as the curved dashed line in figure \ref{fig:age_gradient}).
All stars above this line were removed from the sample.
Subgiants were also removed by eliminating stars brighter than 4th magnitude
in \gaia\ G-band.

The rotation periods of the dwarf stars in the \mct\ sample are shown on a
\gaia\ color-magnitude diagram (CMD) in the top panel of figure
\ref{fig:age_gradient}.
In the bottom panel, the stars are colored by their gyrochronal age,
calculated using the \citet{angus2019} gyrochronology relation.
The stars with old gyrochronal ages, plotted in purple hues, predominantly lie
along the upper edge of the MS, where stellar evolution models predict old
stars to be, however the majority of these `old' stars are bluer than \gcolor\
$\sim$ 1.5 dex.
The lack of gyrochronologically old M dwarfs suggests that either old M dwarfs
are missing from the \mct\ catalog, or the \citet{angus2019} gyrochronology
relation under-predicts the ages of low-mass stars.
Given that lower-mass stars stay active for longer than higher-mass stars
\citep[\eg][]{west2011, kiman2019}, and are therefore more likely to have
measurable rotation periods at old ages, the latter scenario seems likely.
However, it is also possible that the rotation periods of the oldest early M
dwarfs are so long that they are not measurable with Kepler data.
Ground-based rotation period measurements of mid and late M dwarfs indicate
that there is an upper limit to the rotation periods of {\it late} M dwarfs of
around 140 days \citep{newton2016, newton2017, newton2018}, which is much
longer than the longest rotation periods measured in the \mct\ sample (around
70 days).
The apparent lack of old ages for M dwarfs in figure \ref{fig:age_gradient}
may be caused by a combination of ages being underestimated by a poorly
calibrated model, and rotation period detection bias.
% For this reason, it is likely that slowly rotating M dwarfs are missing from
% the \mct\ sample and this could be another reason for the lack of old M
% dwarfs.
The \citet{angus2019} gyrochronology relation is a simple polynomial model,
fit to the period-color relation of Praesepe.
% Since this relation predicts that the oldest stars in the \mct\ sample are
% late-G and early-K dwarfs, it is probably under-predicting the ages of late-K
% and early-M dwarfs.
Inaccuracies are a typical feature of empirically calibrated gyrochronology
models: since there are no (or at least very few) old M dwarfs with rotation
periods, the models are poorly calibrated for these stars.
\begin{figure}
  \caption{
      Top: de-reddened MS \kepler\ stars with \mct\ rotation periods, plotted
    on a \gaia\ CMD.
    We removed photometric binaries and subgiants from the sample by excluding
    stars above the dashed lines.
    Bottom: a zoom-in of the top panel, with stars colored by their
    gyrochronal age \citep{angus2019}, instead of their rotation period.
    A general age gradient is visible across the main sequence.
    Since the \citet{angus2019} relation predicts that the oldest stars in
    the \mct\ sample are late-G and early-K dwarfs, it is probably
    under-predicting the ages of late-K and early-M dwarfs.
}
  \centering
    % \includegraphics[width=1\textwidth]{age_gradient_combo}
    \includegraphics[width=1\textwidth]{CMD_cuts_double}
\label{fig:age_gradient}
\end{figure}

The {\tt Pyia} \citep{price-whelan_2018} and {\tt astropy} \citep{astropy2013,
astropy2018} {\it Python} packages were used to calculate velocities for the
\mct\ sample.
{\tt Pyia} calculates velocity samples from the full \gaia\ uncertainty
covariance matrix via Monte Carlo sampling, thereby accounting for the
covariances between \gaia\ positions, parallaxes and proper motions.
Stars with negative parallaxes, parallax signal-to-noise ratios less than 10,
stars fainter than 16th magnitude, stars with absolute \vb\ uncertainties
greater than 1 \kms\, and stars with galactic latitudes greater than
15\degrees\ (justification provided below) were removed from the sample.
% Because \vb\ is only a close approximation to \vz\ at low galactic latitudes,
% and because latitude is correlated with stellar mass (lower mass stars are
% older and tend to be preferentially located at high $b$, although this trends
% is greatly reduced for stars cooler than 5000 K), we found that excluding
% stars with $b > 15$\degrees\ reduced the number of \vb\ outliers at low
% stellar mass.
% 15\degrees\ was found to be an adequate compromise between maintaining a close
% relationship between \vb\ and \vz\ and retaining a sample large enough to
% provide meaningful results.
% Reducing this cut to 10\degrees\ resulted in a sample size too small to reveal
% trends in the data, and intermediate cuts did not significantly change the
% results.
% We also found that the anti-correlation between stellar mass and $b$ is
% negligable for stars cooler than 5000 K.
% For this reason, and because the rotational evolution of old, G stars does not
% follow a simple Skumanich-like spin-down law \citep{vansaders2016}, we
% restricted our sample to temperatures between \tmin\ and 5000 K, \ie\ K
% dwarfs.

% \subsection{Validating \vb\ dispersion as an age proxy}
% \label{sec:mass-dependent-heating}

% There are two main reasons why \vb\ velocity dispersion may not be a good age
% proxy.
% Firstly, mass-dependent heating may act on the sample, meaning that velocity
% dispersion depends on both age and mass, so cannot be interpreted as a simple
% age proxy.
% Secondly, since stars in the \kepler\ field have a range of galactic
% latitudes, using \vb\ as a stand-in for \vz\ may not be equally valid for all
% stars, and introduce a velocity bias for high latitude stars (which are more
% likely to be cooler and older).
% In this section we demonstrate that neither of these problems seem to be a
% significant issue for our data.

% In order to establish whether \sigmavb\ can be used as an age proxy, we
% searched for signs of mass-dependent heating within the \kepler\ field.
% Mass-dependent dynamical heating may result from lower-mass stars experiencing
% greater velocity changes when gravitationally perturbed than more massive
% stars.
% It has not been unambiguously observed in the galactic disk because of the
% strong anti-correlation between stellar mass and stellar age.
% Less massive stars do indeed have larger velocity dispersions, however they
% are also older on average.
% This mass-age degeneracy is highly reduced in M dwarfs because their
% main-sequence lifetimes are longer than the age of the Universe, and no
% evidence for mass-dependent heating has previously been found in M dwarfs
% \citep[\eg][]{faherty2009, newton2016}.

% To investigate whether mass-dependent heating could be acting on the \kepler\
% sample, we selected late K and M dwarfs observed by both \kepler\ and \gaia,
% whose MS lifetimes exceed around 11 Gyrs and are therefore representative of
% the initial mass function.
% We could not perform this analysis on the \mct\ sample, because it only
% includes stars with {\it detectable} rotation periods, and since lower-mass
% stars stay active for longer it is likely that it contains a strong mass-age
% correlation.
% We selected all \kepler\ targets with dereddened \gaia\ \gcolor\ colors
% greater than 1.2 (corresponding to an effective temperature $\lesssim$
% 4800 K) and absolute \gaia\ $G$-band magnitudes $<$ 4.
% We also eliminated visual binaries by removing stars above a 6th order
% polynomial, fit to the MS on the \gaia\ CMD (similar to the one shown in
% figure \ref{fig:age_gradient}).
% We then applied the quality cuts described above in section
% \ref{sec:the_data}.
% To search for evidence of mass-dependent heating we calculated the (\vb)
% velocity dispersion of stars in effective temperature bins.
% Sigma clipping was performed at the 3$\sigma$ level to remove high velocity
% outliers before calculating the standard deviation of stars in each bin.
% These high velocity outliers may be very old late K and M dwarfs, or they
% result from using \vb\ instead of \vz, which introduces additional velocity
% scatter.

% Figure \ref{fig:vb_vs_teff} shows velocity and velocity dispersion as a
% function of effective temperature \footnote{calculated by transforming
% dereddened \gaia\ colors using equation \ref{eq:curtis}.} for the K and M
% \kepler\ dwarf sample.
% Velocity dispersion very slightly {\it decreases} with decreasing temperature,
% the opposite of the trend expected for mass-dependent heating, however the
% slope is only inconsistent with zero at the 1.3 $\sigma$ level.
% This trend may be due to a selection bias: cooler stars are fainter and
% therefore typically closer, with smaller heights above the galactic plane and
% smaller velocities.
% The essential point however, is that we do not see evidence for mass-dependent
% heating acting on stars in the \kepler\ field, indicating that velocity
% dispersion {\it can} be used as an age proxy.
% This analysis was performed using \vb\ but we also examined the {\it vertical}
% velocities of the 537 stars in this sample with RV measurements.
% Again, no evidence was found for mass-dependent heating: the slope of the
% velocity dispersion-temperature relation was consistent with zero.
% \begin{figure}
%   \caption{
%       Top: Stellar velocity (\vb) as a function of \teff\ for
%       \kepler\ K and M dwarfs.
% Vertical lines indicate different \teff-groupings used to calculate velocity
%     dispersion.
% Pink stars were not included in velocity dispersion calculations as they were
%     either removed as outliers during a sigma clipping process, or they lie at
%     the sparcely populated, extremely cool end of the temperature range.
%     Velocity dispersion and \teff\ are slightly positively correlated, likely
%     due to a brightness-related selection bias, indicating that mass-dependent
%     heating does not significantly affect low-mass stars in the \kepler\
%     field.
% }
%   \centering
%     \includegraphics[width=1\textwidth]{vb_vs_teff}
% \label{fig:vb_vs_teff}
% \end{figure}

% Having found no strong evidence for mass-dependent heating, we next tested
% the validity of \vb\ as a proxy for \vz\ in more detail.
% At a galactic latitude, $b$, of zero, $v_b=v_z$, however for increasing values
% of $b$, this equivalence becomes an approximation that grows noisier with $b$.
% To test the validity of the \vb$\sim$\vz\ approximation over a range of
% latitudes we downloaded stellar data from the \Gaia\ Universe Model Snapshot
% (GUMS) simulation -- a simulated \Gaia\ catalog \citep{robin2012}.
% We downloaded stars from four pointings in the \kepler\ field with galactic
% latitudes of around 5\degrees, 10\degrees, 15\degrees, and 20\degrees, out to
% a limiting magnitude of 16 dex, and calculated their \vz\ and \vb\ velocities.
% The relationship between \vz\ and \vb\ is close to 1:1, with \vz\ greater than
% \vb\ by around 4.5 kms$^{-1}$ at $b=5$, due to the Sun's own motion in the
% Galaxy.
% We subtracted this offset and examined the residuals of the \vz\ -- \vb\
% relationship to investigate the variance as a function of Galactic latitude.
% We found that \vb\ is drawn from a heavy-tailed distribution, centered on \vz,
% with standard deviation increasing with $b$ (see figure \ref{fig:vb_vz}).
% The standard deviation of \vz-\vb\ was around 3kms$^{-1}$ at $b \sim 5^\circ
% $, 4kms$^{-1}$ at 10$^\circ$, 6kms$^{-1}$ at 15$^\circ$, and 9kms$^{-1}$ at
% 20$^\circ$.

% % Figure \ref{fig:vb_vz} shows the differences between \vz\ and \vb\ velocities
% % over different Galactic latitudes.
% % Kernel Density Estimates (KDEs) are shown as solid black lines and Gaussian
% % fits are shown as dashed blue lines.
% % The \vb\-\vz\ residuals are close to Gaussian, with slightly heavy tails, and
% % the variance increases with increasing Galactic latitude.
% \begin{figure}
%   \caption{
% This figure demonstrates the variance in the relationship between \vb\ and
%     \vz\ for stars in the \kepler\ field, based on the GUMS simulation.
% The panels show a kernel density estimator (KDE) (black solid line) for
%     the \vz -- \vb\ residuals of stars in the GUMS simulation at four
%     different Galactic latitudes.
% Blue dashed lines show Gaussian fits to these KDEs.
% The distributions are close to Gaussian, with slightly heavy tails.
% The standard deviations of the Gaussian fits increase with Galactic latitude.
% This figure illustrates how using \vb\ instead of \vz\ artificially
%     increases velocity dispersion, especially at high latitudes.
% }
%   \centering
%     \includegraphics[width=1\textwidth]{vb_vz}
% \label{fig:vb_vz}
% \end{figure}

% Since we are concerned with velocity {\it dispersions}, rather than velocities
% themselves, we also compared the \vb\ and \vz\ velocity dispersions as a
% function of temperature for
% stars downloaded from the GUMS simulation.
% For stars at galactic latitudes of 15\degrees\ or less, \sigmavb\ was
% consistent with $\sigma_{v{\bf z}}$, within uncertainties, however, at higher
% latitudes the two quantities became significantly different.
% For this reason we proceeded by only including stars with galactic latitudes
% less than 15\degrees\ in our analysis.
% Although it seems that the transformation between \vz\ and \vb\ does not {\it
% strongly} affect our results, we cannot rule out the possibility that it
% introduces systematic biases into the velocity dispersions we present here.
% In \gaia\ DR3, RVs will be available for most stars in this sample, providing
% an opportunity to validate (or correct) the results presented here.

% % Since we use \vb, not \vz\ in our analysis, it is possible that the sample
% % selection function could influence our results.
% % For example, higher-mass stars tend to be younger and reside closer to the
% % galactic disk mid-plane (low galactic latitudes).
% % Since \vb\ is only similar to \vz\ at low latitudes, this could mean that
% % lower-mass stars typically have greater velocity dispersions due to .
% % \teff\ and galactic latitude, $b$,
% % could result in
% % Stars at higher latitudes have additional velocity components in the ${\bf x}$
% % and ${\bf y}$ directions, which could increase \vb\ but not \vz.
% % Again however, since the relationship between \sigmavb\ and \teff\ is
% % positively, not negatively correlated for cool stars in the \kepler\ field,
% % this effect is probably too small to influence our results.

% % Figure \ref{fig:vb_vs_teff} indicates that mass-dependent heating does not
% % strongly affect the \mct\ sample of rotating \kepler\ stars.
% % For this reason, we assume that age difference is the major cause of velocity
% % dispersion differences between groups.
% % In other words, (\vb) velocity dispersion is a reliable age proxy for the
% % \mct\ sample.

% Because of the noisy relationship between \vb\ and \vz\, in this paper we do
% not attempt to convert velocity dispersion (\sigmavb) into an age via an
% age-velocity dispersion relation (AVR) \citep[\eg][]{holmberg2009}.
% Although it seems \sigmavb\ can be used to roughly rank stars by age, a more
% careful analysis that includes formal modeling of the \vb\ -- \vz\
% relationship will be needed to calculate absolute ages.
% % For example, the velocity distributions could be modeled as a mixture of
% % Gaussians in order to account for the additional velocity dispersion caused by
% % the \vz-\vb\ transformation.
% The RVs of most of these stars will become available in \Gaia\ DR3, allowing
% calculations of \vz, which can be used to calculate more reliable ages via an
% AVR.


% % We then selected groups of stars within different gyrochronal age ranges and
% % calculated the standard deviation of \vb\ velocities (\sigmavb), as a function
% % of effective temperature for each age group.
% % Ages were calculated using dereddened \gaia\ \gcolor\ color, however
% % throughout this paper we show rotation periods as a function of effective
% % temperature, \teff, because it is easier to divide stars into bins of roughly
% % equal numbers in \teff-space than in color-space.
