% 270 words
% Limit: 500 words
\section{Method}

\subsection{Calculating velocites}

We crossmatched the \mct\ catalog of stellar rotation periods, measured from
\kepler\ light curves with the \gaia\ DR2 catalog.
We added 2MASS K-band photometry from the Kepler Input Catalog
\citep{brown2011} and SDSS g-band photometry \racomment{(citations)}.

\begin{figure}
  \caption{
A color magnitude diagram in SDSS g and 2MASS K bands showing the
\citet{mcquillan2014} sample, colored by their rotation periods.
We excluded photometric binaries and subgiants from our analysis by removing
stars above the two dashed lines.
The rotation periods of binaries and subgiants do not follow a Skumanich-like
braking law and were removed from our sample.
% since we rely on a simple scaling
% between rotation period and age to make the argument that the rotation period
% is not caused by incorrect period measurements.
}
  \centering
    \includegraphics[width=1\textwidth]{CMD_cuts}
\label{fig:CMD_cuts}
\end{figure}

In order to explore the age of this stellar population from a rotation-period
standpoint, it was first necessary to remove visual binaries and subgiants
from the sample.
The rotation-period evolution of these two types of stars is generally
different to that of single stars which more usually follow a Skumanich-like
spin-down law.
Tidal and magnetic interactions between the two components of a binary system
can influence the rotation periods of both stars, and the expanding envelopes
of subgiants drive rapid spin-down through conservation of angular momentum.
We removed visual binaries from the sample by fitting a 6th-order polynomial
to the main sequence and raising it by subtracting 0.15 magnitudes to
approximate the line between the single-star and binary main sequences.
This polynomial is shown in figure \ref{fig:CMD_cuts}.
\racomment{Why did we do this in g-K?}
In order to remove subgiants, we removed stars brighter than 6th magnitude in
the g-band from the sample.

We used the {\tt Pyia} {\it Python} package to calculate stellar velocities.
{\tt Pyia} has built-in functionality for calculating velocity samples from
the full \gaia\ uncertainty covariance matrix via Monte Carlo sampling.
It therefore not only incorporates uncertainties on the \gaia\ positions
parallaxes and proper motions, it also accounts for the covariance between
these properties.
We calculated \vb\ for each star in the \mct\ catalog that survived the visual
binary and subgiant cuts.

% \begin{figure}
%   \caption{
% the rotation periods of stars in the \citet{mcquillan2014} sample vs.
% effective temperature.
% black points are dwarf stars identified as non-photometric binaries, hotter
% than 4800 k.
% blue circle points are non-photometric binary dwarfs, cooler than 4800 k, with
% a rotation period and \gaia\ color indicating they are older than 1.1 gyr.
% orange squares are stars that with rotation periods that fall just below the
% gap: they have rotation-ages between 0.5 and 1.1 gyrs.
% green triangles are stars with rotation periods faster than the main envelop
% of stars.
% these are probably binaries whose rotation periods are synchronized to their
% orbits and have been spun-up via tidal interactions.
% }
%   \centering
%     \includegraphics[width=1\textwidth]{period_teff}
% \label{fig:period_teff}
% \end{figure}
