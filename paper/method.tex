% 270 words
% Limit: 500 words

\section{Method}
\label{sec:method}

\subsection{The data}
\label{sec:the_data}

We used the publicly available \kepler-\gaia\ DR2 crossmatched
catalog\footnote{Available at gaia-kepler.fun} to combine the \mct\ catalog of
stellar rotation periods, measured from \kepler\ light curves, with the \gaia\
DR2 catalog of parallaxes, proper motions and apparent magnitudes.
Reddening and extinction from dust was calculated for each star using the
Bayestar dust map implemented in the {\tt dustmaps} {\it Python} package
\citep{green2018}, and {\tt astropy} \citep{astropy2013}.
We estimated effective temperatures from dereddened \Gaia\ \gcolor\ color,
using an 8th-order polynomial relation calibrated using .... stars
\racomment{ask Jason for details}.
\begin{equation}
    \mathrm{T_{eff}} = 8960 -4802C + 1931C^2 -2446C^3 + 2669C^4 - 1324C^5 +
    301C^6 - 26C^7,
% 8959.8112335205078, -4801.5566310882568, 1931.4756631851196,
%           -2445.9980716705322, 2669.0248055458069, -1324.0671020746231,
%           301.13205924630165, -25.923997443169355]
\label{eq:curtis}
\end{equation}
where C is \gaia\ \gcolor.

Visual binaries and subgiants were removed from the sample by applying cuts to
the color-magnitude diagram (CMD), shown in figure \ref{fig:age_gradient}.
A 6th-order polynomial was fit to the main sequence and raised by 0.27 dex to
approximate the division between single stars and visual binaries.
All stars above this line were removed from the sample.
Subgiants were also removed by eliminating stars brighter than 6th magnitude
in \gaia\ G-band.

The dwarf stars in the \mct\ sample are shown on a \gaia\ color-magnitude
diagram (CMD) in figure \ref{fig:age_gradient}.
The stars are colored by their gyrochronal age, calculated using the
\citet{angus2019} gyrochronology relation.
The stars with old gyrochronal ages, plotted in yellow hues, predominantly lie
along the upper edge of the MS, where stellar evolution models predict old
stars to be, however the majority of these `old' stars are bluer than \gcolor\
$\sim$ 1.5 dex.
The lack of chronologically old M dwarfs suggests that either old M dwarfs are
missing from the \mct\ catalog, or the \citet{angus2019} gyrochronology
relation under-predicts the ages of low-mass stars.
Given that lower-mass stars stay active for longer than higher-mass stars
\citep[\eg][]{west2011}, and are therefore more likely to have measurable
rotation periods at old ages, the latter scenario seems more likely.
The \citet{angus2019} gyrochronology relation is a simple polynomial model,
fit to the period-color relation of Praesepe.
Since this relation predicts that the oldest stars in the \mct\ sample are
late-G and early-K dwarfs, it is probably under-predicting the ages of late-K
and early-M dwarfs.
This is a common problem with gyrochronology models: since suitable calibrator
stars are small in number, models cannot be completely calibrated across a
broad range of parameter space.
Plotting gyrochronal ages on a color-magnitude diagram reveals the
deficiencies in the \citet{angus2019} model used to calculate them, however
imperfections are expected given that no published gyrochronology model is
thought to be perfectly calibrated.
\begin{figure}
  \caption{
Dereddened MS \kepler\ stars with \mct\ rotation periods on the \gaia\ CMD.
We excluded visual binaries by removing stars above the dashed line.
    Points are colored by their gyrochronal
    age, according to the
    \citet{angus2019} gyrochronology relation.
    A general age gradient is visible across the main sequence.
    Since the \citet{angus2019} relation predicts that the oldest stars in
    the \mct\ sample are late-G and early-K dwarfs, it is probably
    under-predicting the ages of late-K and early-M dwarfs.
}
  \centering
    \includegraphics[width=1\textwidth]{age_gradient_combo}
\label{fig:age_gradient}
\end{figure}

The {\tt Pyia} \citep{price-whelan_2018} and {\tt astropy} \citep{astropy2013,
astropy2018} {\it Python} packages were used to calculate stellar velocities.
{\tt Pyia} calculates velocity samples from the full \gaia\ uncertainty
covariance matrix via Monte Carlo sampling.
It therefore not only incorporates uncertainties on the \gaia\ positions
parallaxes and proper motions, it also accounts for the {\it covariance}
between these properties.
Stars with negative parallaxes, parallax signal-to-noise ratios less than 10,
stars fainter than 16th magnitude, stars with absolute \vb\ uncertainties
greater than 1 \kms\, and stars with galactic latitudes greater than
15\degrees\ (justification provided below) were removed from the sample.
% Because \vb\ is only a close approximation to \vz\ at low galactic latitudes,
% and because latitude is correlated with stellar mass (lower mass stars are
% older and tend to be preferentially located at high $b$, although this trends
% is greatly reduced for stars cooler than 5000 K), we found that excluding
% stars with $b > 15$\degrees\ reduced the number of \vb\ outliers at low
% stellar mass.
% 15\degrees\ was found to be an adequate compromise between maintaining a close
% relationship between \vb\ and \vz\ and retaining a sample large enough to
% provide meaningful results.
% Reducing this cut to 10\degrees\ resulted in a sample size too small to reveal
% trends in the data, and intermediate cuts did not significantly change the
% results.
% We also found that the anti-correlation between stellar mass and $b$ is
% negligable for stars cooler than 5000 K.
% For this reason, and because the rotational evolution of old, G stars does not
% follow a simple Skumanich-like spin-down law \citep{vansaders2016}, we
% restricted our sample to temperatures between \tmin\ and 5000 K, \ie\ K
% dwarfs.

\subsection{Validating \vb\ dispersion as an age proxy}
\label{sec:mass-dependent-heating}

There are two main reasons why \vb\ velocity dispersion may not be a good age
proxy.
Firstly, mass-dependent heating may act on the sample, meaning that velocity
dispersion depends on both age and mass, so cannot be interpreted as a simple
age proxy.
Secondly, since stars in the \kepler\ field have a range of galactic
latitudes, using \vb\ as a stand-in for \vz\ may not be equally valid for all
stars, and introduce a velocity bias for high latitude stars (which are more
likely to be cooler and older).
In this section we demonstrate that neither of these problems seem to be a
significant issue for our data.

If lower-mass stars experience greater velocity changes when gravitationally
perturbed, and are dynamically heated more efficiently than higher-mass stars,
velocity dispersion would be a function of both age {\it and} mass and cannot
be interpreted as a straightforward age proxy.
So, in order to establish whether \sigmavb\ can be used as an age proxy, we
searched for signs of mass-dependent heating within the \kepler\ field.
Mass-dependent dynamical heating has not been unambiguously observed in the
galactic disk because of the strong anti-correlation between stellar mass and
stellar age.
Less massive stars do indeed have larger velocity dispersions, however they
are also older on average.
This mass-age degeneracy is highly reduced in M dwarfs because their
main-sequence lifetimes are longer than the age of the Universe, and no
evidence for mass-dependent heating has previously been found in M dwarfs
\citep[\eg][]{faherty2009, newton2016}.

To further investigate whether mass-dependent heating could be acting on the
\kepler\ sample, we selected late K and M dwarfs observed by both \kepler\ and
\gaia, whose MS lifetimes exceed around 11 Gyrs and are therefore
representative of the initial mass function\footnote{We could not perform this
analysis on the \mct\ sample, because it only includes stars with {\it
detectable} rotation periods, and since lower-mass stars stay active for
longer it is likely that it contains a strong mass-age correlation.}.
We selected all \kepler\ targets with dereddened \gaia\ \gcolor\ colors
greater than 1.2 (corresponding to an effective temperature $\lesssim$
4800 K) and absolute \gaia\ $G$-band magnitudes $<$ 4.
We also eliminated visual binaries by removing stars above a 6th order
polynomial, fit to the MS on the \gaia\ CMD (similar to the one shown in
figure \ref{fig:age_gradient}).
We then applied the quality cuts described above in section
\ref{sec:the_data}.
To search for evidence of mass-dependent heating we calculated the (\vb)
velocity dispersion of stars in effective temperature bins.
Sigma clipping was performed at the 3$\sigma$ level to remove high velocity
outliers before calculating the standard deviation of stars in each bin.
These high velocity outliers may be very old late K and M dwarfs, or they
result from using \vb\ instead of \vz, which introduces additional velocity
scatter.

Figure \ref{fig:vb_vs_teff} shows velocity and velocity dispersion as a
function of effective temperature \footnote{calculated by transforming
dereddened \gaia\ colors using equation \ref{eq:curtis}.} for the K and M
\kepler\ dwarf sample.
Velocity dispersion very slightly {\it decreases} with decreasing temperature,
the opposite of the trend expected for mass-dependent heating, however the
slope is only inconsistent with zero at the 1.3 $\sigma$ level.
This trend may be due to a selection bias: cooler stars are fainter and
therefore typically closer, with smaller heights above the galactic plane and
smaller velocities.
The essential point however, is that we do not see evidence for mass-dependent
heating acting on stars in the \kepler\ field, indicating that velocity
dispersion {\it can} be used as an age proxy.
This analysis was performed using \vb\ but we also examined the {\it vertical}
velocities of the 537 stars in this sample with RV measurements.
Again, no evidence was found for mass-dependent heating: the slope of the
velocity dispersion-temperature relation was consistent with zero.
\begin{figure}
  \caption{
      Top: Stellar velocity (\vb) as a function of \teff\ for
      \kepler\ K and M dwarfs.
Vertical lines indicate different \teff-groupings used to calculate velocity
    dispersion.
Pink stars were not included in velocity dispersion calculations as they were
    either removed as outliers during a sigma clipping process, or they lie at
    the sparcely populated, extremely cool end of the temperature range.
    Velocity dispersion and \teff\ are slightly positively correlated, likely
    due to a brightness-related selection bias, indicating that mass-dependent
    heating does not significantly affect low-mass stars in the \kepler\
    field.
}
  \centering
    \includegraphics[width=1\textwidth]{vb_vs_teff}
\label{fig:vb_vs_teff}
\end{figure}

Having found no strong evidence for mass-dependent heating, we next tested
the validity of \vb\ as a proxy for \vz\ in more detail.
At a galactic latitude, $b$, of zero, $v_b=v_z$, however for increasing values
of $b$, this equivalence becomes an approximation that grows noisier with $b$.
To test the validity of the \vb$\sim$\vz\ approximation over a range of
latitudes we downloaded stellar data from the \Gaia\ Universe Model Snapshot
(GUMS) simulation -- a simulated \Gaia\ catalog \citep{robin2012}.
We downloaded stars from four pointings in the \kepler\ field with galactic
latitudes of around 5\degrees, 10\degrees, 15\degrees, and 20\degrees, out to
a limiting magnitude of 16 dex, and calculated their \vz\ and \vb\ velocities.
The relationship between \vz\ and \vb\ is close to 1:1, with \vz\ greater than
\vb\ by around 4.5 kms$^{-1}$ at $b=5$, due to the Sun's own motion in the
Galaxy.
We subtracted this offset and examined the residuals of the \vz\ -- \vb\
relationship to investigate the variance as a function of Galactic latitude.
We found that \vb\ is drawn from a heavy-tailed distribution, centered on \vz,
with standard deviation increasing with $b$ (see figure \ref{fig:vb_vz}).
The standard deviation of \vz-\vb\ was around 3kms$^{-1}$ at $b \sim 5^\circ
$, 4kms$^{-1}$ at 10$^\circ$, 6kms$^{-1}$ at 15$^\circ$, and 9kms$^{-1}$ at
20$^\circ$.

% Figure \ref{fig:vb_vz} shows the differences between \vz\ and \vb\ velocities
% over different Galactic latitudes.
% Kernel Density Estimates (KDEs) are shown as solid black lines and Gaussian
% fits are shown as dashed blue lines.
% The \vb\-\vz\ residuals are close to Gaussian, with slightly heavy tails, and
% the variance increases with increasing Galactic latitude.
\begin{figure}
  \caption{
This figure demonstrates the variance in the relationship between \vb\ and
    \vz\ for stars in the \kepler\ field, based on the GUMS simulation.
The panels show a kernel density estimator (KDE) (black solid line) for
    the \vz -- \vb\ residuals of stars in the GUMS simulation at four
    different Galactic latitudes.
Blue dashed lines show Gaussian fits to these KDEs.
The distributions are close to Gaussian, with slightly heavy tails.
The standard deviations of the Gaussian fits increase with Galactic latitude.
This figure illustrates how using \vb\ instead of \vz\ artificially
    increases velocity dispersion, especially at high latitudes.
}
  \centering
    \includegraphics[width=1\textwidth]{vb_vz}
\label{fig:vb_vz}
\end{figure}

Since we are concerned with velocity {\it dispersions}, rather than velocities
themselves, we also compared the \vb\ and \vz\ velocity dispersions as a
function of temperature for
stars downloaded from the GUMS simulation.
For stars at galactic latitudes of 15\degrees\ or less, \sigmavb\ was
consistent with $\sigma_{v{\bf z}}$, within uncertainties, however, at higher
latitudes the two quantities became significantly different.
For this reason we proceeded by only including stars with galactic latitudes
less than 15\degrees\ in our analysis.
Although it seems that the transformation between \vz\ and \vb\ does not {\it
strongly} affect our results, we cannot rule out the possibility that it
introduces systematic biases into the velocity dispersions we present here.
In \gaia\ DR3, RVs will be available for most stars in this sample, providing
an opportunity to validate (or correct) the results presented here.

% Since we use \vb, not \vz\ in our analysis, it is possible that the sample
% selection function could influence our results.
% For example, higher-mass stars tend to be younger and reside closer to the
% galactic disk mid-plane (low galactic latitudes).
% Since \vb\ is only similar to \vz\ at low latitudes, this could mean that
% lower-mass stars typically have greater velocity dispersions due to .
% \teff\ and galactic latitude, $b$,
% could result in
% Stars at higher latitudes have additional velocity components in the ${\bf x}$
% and ${\bf y}$ directions, which could increase \vb\ but not \vz.
% Again however, since the relationship between \sigmavb\ and \teff\ is
% positively, not negatively correlated for cool stars in the \kepler\ field,
% this effect is probably too small to influence our results.

% Figure \ref{fig:vb_vs_teff} indicates that mass-dependent heating does not
% strongly affect the \mct\ sample of rotating \kepler\ stars.
% For this reason, we assume that age difference is the major cause of velocity
% dispersion differences between groups.
% In other words, (\vb) velocity dispersion is a reliable age proxy for the
% \mct\ sample.

Because of the noisy relationship between \vb\ and \vz\, in this paper we do
not attempt to convert velocity dispersion (\sigmavb) into an age via an
age-velocity dispersion relation (AVR) \citep[\eg][]{holmberg2009}.
Although it seems \sigmavb\ can be used to roughly rank stars by age, a more
careful analysis that includes formal modeling of the \vb\ -- \vz\
relationship will be needed to calculate absolute ages.
% For example, the velocity distributions could be modeled as a mixture of
% Gaussians in order to account for the additional velocity dispersion caused by
% the \vz-\vb\ transformation.
The RVs of most of these stars will become available in \Gaia\ DR3, allowing
calculations of \vz, which can be used to calculate more reliable ages via an
AVR.


% We then selected groups of stars within different gyrochronal age ranges and
% calculated the standard deviation of \vb\ velocities (\sigmavb), as a function
% of effective temperature for each age group.
% Ages were calculated using dereddened \gaia\ \gcolor\ color, however
% throughout this paper we show rotation periods as a function of effective
% temperature, \teff, because it is easier to divide stars into bins of roughly
% equal numbers in \teff-space than in color-space.
