\section{Method}
\label{sec:method}

\subsection{The data}
\label{sec:the_data}

We used the publicly available \kepler-\gaia\ DR2 crossmatched
catalog\footnote{Available at gaia-kepler.fun} to combine the \mct\ catalog of
stellar rotation periods, measured from \kepler\ light curves, with the \gaia\
DR2 catalog of parallaxes, proper motions and apparent magnitudes.
Reddening and extinction from dust was calculated for each star using the
Bayestar dust map implemented in the {\tt dustmaps} {\it Python} package
\citep{green2018}, and {\tt astropy} \citep{astropy2013, astropy2018}.

For this work, we used the precise \textit{Gaia} DR2 photometric color,
$G_{\rm BP} - G_{\rm RP}$, to estimate \teff\ for the Kepler rotators.
To calibrate this relation, Curtis \etal\ (2020, in prep) combined effective
temperature measurements for nearby, unreddened field stars in benchmark
samples, including FGK stars characterized with high-resolution optical
spectroscopy \citep{brewer2016}, M dwarfs characterized with low-resolution
optical and near-infrared spectroscopy \citep{mann2015}, and K and M dwarfs
characterized with interferometry and bolometric flux analyses
\citep{boyajian2012}.
This empirical color--temperature relation is valid over the color range $0.55
< (G_{\rm BP} - G_{\rm RP})_0 < 3.20$, corresponding to $6470 < T_{\rm eff} <
3070$~K.
The dispersion about the relation implies a high precision of 50~K.
These benchmark data enable us to accurately estimate \teff\ for cool dwarfs
\citep[\eg][]{rabus2019}, and allows us to correct for interstellar reddening
at all temperatures\footnote{The color--temperature relation is described in
detail in the Appendix of, and the formula is provided in Table 4 of, Curtis
\etal\ (2020, in prep).}.
The equation we used to calculate photometric temperatures from Gaia \gcolor\
color is a seventh-order polynomial with coefficients given in table
\ref{tab:coeffs}.
\begin{table}[h!]
  \begin{center}
      \caption{
          Coefficient values for the 7th-order polynomial used to estimate
      \teff\ from \Gaia\ \gcolor\ color, calibrated in Curtis \etal\ (2020, in
      prep).}
    \label{tab:coeffs}
    \begin{tabular}{l|c} % <-- Alignments: 1st column left and 2nd middle, with vertical lines in between
        (\gcolor\ ) exponent & Coefficient  \\
      \hline
      $0$ & -416.585 \\
      $1$ & 39780.0  \\
      $2$ & -84190.5 \\
      $3$ & 85203.9  \\
      $4$ & -48225.9 \\
      $5$ & 15598.5  \\
      $6$ & -2694.76 \\
      $7$ & 192.865  \\
    \end{tabular}
  \end{center}
\end{table}

Photometric binaries and subgiants were removed from the \mct\ sample by
applying cuts to the color-magnitude diagram (CMD), shown in figure
\ref{fig:age_gradient}.
A 6th-order polynomial was fit to the main sequence and raised by 0.27 dex to
approximate the division between single stars and photometric binaries (shown
as the curved dashed line in figure \ref{fig:age_gradient}).
All stars above this line were removed from the sample.
Potential subgiants were also removed by eliminating stars brighter than 4th
absolute magnitude in \gaia\ G-band.
This cut also removed a number of main sequence F stars from our sample,
however these hot stars are not the focus of our gyrochronology study since
their small convective zones inhibit the generation of a strong magnetic
field.
The removal of photometric binaries and evolved/hot stars reduced the total
sample of around 34,000 stars by almost 10,000.

The rotation periods of the dwarf stars in the \mct\ sample are shown on a
\gaia\ color-magnitude diagram (CMD) in the top panel of figure
\ref{fig:age_gradient}.
In the bottom panel, the stars are colored by their gyrochronal age,
calculated using the \citet{angus2019} gyrochronology relation.
The stars with old gyrochronal ages, plotted in purple hues, predominantly lie
along the upper edge of the MS, where stellar evolution models predict old
stars to be, however the majority of these `old' stars are bluer than \gcolor\
$\sim$ 1.5 dex.
The lack of gyrochronologically old M dwarfs suggests that either old M dwarfs
are missing from the \mct\ catalog, or the \citet{angus2019} gyrochronology
relation under-predicts the ages of low-mass stars.
Given that lower-mass stars stay active for longer than higher-mass stars
\citep[\eg][]{west2008, newton2017, kiman2019}, and are therefore more likely
to have measurable rotation periods at old ages, the latter scenario seems
likely.
However, it is also possible that the rotation periods of the oldest early M
dwarfs are so long that they are not measurable with Kepler data.
Ground-based rotation period measurements of mid and late M dwarfs indicate
that there is an upper limit to the rotation periods of {\it late} M dwarfs of
around 140 days \citep{newton2016, newton2018}, which is much
longer than the longest rotation periods measured in the \mct\ sample (around
70 days).
The apparent lack of old gyro-ages for M dwarfs in figure
\ref{fig:age_gradient} may be caused by a combination of ages being
underestimated by a poorly calibrated model, and rotation period detection
bias.
The \citet{angus2019} gyrochronology relation is a simple polynomial model,
fit to the period-color relation of Praesepe.
Inaccuracies at low masses are a typical feature of empirically calibrated
gyrochronology models since there are no (or at least very few) old M dwarfs
with rotation periods and the models are poorly calibrated for these stars.
\begin{figure}
  \caption{
      Top: de-reddened MS \kepler\ stars with \mct\ rotation periods, plotted
    on a \gaia\ CMD.
    We removed photometric binaries and subgiants from the sample by excluding
    stars above the dashed lines.
    Bottom: a zoom-in of the top panel, with stars colored by their
    gyrochronal age \citep{angus2019}, instead of their rotation period.
    A general age gradient is visible across the main sequence.
    Since the \citet{angus2019} relation predicts that the oldest stars in
    the \mct\ sample are late-G and early-K dwarfs, it is probably
    under-predicting the ages of late-K and early-M dwarfs.
}
  \centering
    \includegraphics[width=1\textwidth]{CMD_cuts_double}
\label{fig:age_gradient}
\end{figure}

The {\tt Pyia} \citep{price-whelan_2018} and {\tt astropy} \citep{astropy2013,
astropy2018} {\it Python} packages were used to calculate velocities for the
\mct\ sample.
{\tt Pyia} calculates velocity samples from the full \gaia\ uncertainty
covariance matrix via Monte Carlo sampling, thereby accounting for the
covariances between \gaia\ positions, parallaxes and proper motions.
Stars with negative parallaxes or parallax signal-to-noise ratios less than 10
(around 3,000 stars), stars fainter than 16th magnitude (200 stars), stars
with absolute \vb\ uncertainties greater than 1 \kms\ (1000 stars), and stars
with galactic latitudes greater than 15\degrees\ (5500 stars, justification
provided in the appendix) were removed from the sample.
Finally, we removed almost 2000 stars with rotation periods shorter than the
main population of periods, since this area of the period-\teff\ diagram is
sparsely populated.
We removed these rapid rotators by cutting out stars with gyrochronal ages
less than 0.5 Gyr \citep[based on the][gyro-model]{angus2019}, because a 0.5
Gyr gyrochrone\footnote{A gyrochrone is a gyrochronological isochrone, or a
line of constant age in period-\teff, or period-color space.} traces the
bottom edge of the main population of rotation periods.
After these cuts, around 13,000 stars out of the original $\sim$34,000 were
included in the sample.
