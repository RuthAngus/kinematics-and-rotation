% 270 words
% Limit: 500 words

\section{Method}
\label{sec:method}

\subsection{The data}
\label{sec:the_data}

We used the publicly available \kepler-\gaia\ DR2 crossmatched
catalog\footnote{Available at gaia-kepler.fun} to combine the \mct\ catalog of
stellar rotation periods, measured from \kepler\ light curves, with the \gaia\
DR2 catalog of parallaxes, proper motions and apparent magnitudes.
Reddening and extinction from dust was calculated for each star using the
Bayestar dust map implemented in the {\tt dustmaps} {\it Python} package
\citep{green2018}, and {\tt astropy} \citep{astropy2013}.
We estimated effective temperatures from dereddened \Gaia\ \gcolor\ color,
using an 8th-order polynomial relation calibrated using .... stars
\racomment{ask Jason for details}.
\begin{equation}
    \mathrm{T_{eff}} = 8960 -4802C + 1931C^2 -2446C^3 + 2669C^4 - 1324C^5 +
    301C^6 - 26C^7,
% 8959.8112335205078, -4801.5566310882568, 1931.4756631851196,
%           -2445.9980716705322, 2669.0248055458069, -1324.0671020746231,
%           301.13205924630165, -25.923997443169355]
\label{eq:curtis}
\end{equation}
where C is \gaia\ \gcolor.

Visual binaries and subgiants were removed from the sample by applying cuts to
the color-magnitude diagram (CMD), shown in figure \ref{fig:age_gradient}.
A 6th-order polynomial was fit to the main sequence and raised by 0.27 dex to
approximate the division between single stars and visual binaries.
All stars above this line were removed from the sample.
Subgiants were also removed by eliminating stars brighter than 6th magnitude
in \gaia\ G-band.
\begin{figure}
  \caption{
Dereddened MS \kepler\ stars with \mct\ rotation periods on the \gaia\ CMD.
We excluded visual binaries by removing stars above the dashed line.
    Points are colored by their gyrochronal
    age, according to the
    \citet{angus2019} gyrochronology relation.
    A general age gradient is visible across the main sequence.
}
  \centering
    \includegraphics[width=1\textwidth]{age_gradient_combo}
\label{fig:age_gradient}
\end{figure}

The {\tt Pyia} \citep{price-whelan_2018} and {\tt astropy} \citep{astropy2013,
astropy2018} {\it Python} packages were used to calculate stellar velocities.
{\tt Pyia} calculates velocity samples from the full \gaia\ uncertainty
covariance matrix via Monte Carlo sampling.
It therefore not only incorporates uncertainties on the \gaia\ positions
parallaxes and proper motions, it also accounts for the {\it covariance}
between these properties.
Stars with negative parallaxes, parallax signal-to-noise ratios less than 10,
stars fainter than 16th magnitude, stars with absolute \vb\ uncertainties
greater than 1 \kms\, and stars with galactic latitudes greater than
15\degrees\ were removed from the sample.
Because \vb\ is only a close approximation to \vz\ at low galactic latitudes,
and because latitude is correlated with stellar mass (lower mass stars are
older and tend to be preferentially located at high $b$, although this trends
is greatly reduced for stars cooler than 5000 K), we found that excluding
stars with $b > 15$\degrees\ reduced the number of \vb\ outliers at low
stellar mass.
15\degrees\ was found to be an adequate compromise between maintaining a close
relationship between \vb\ and \vz\ and retaining a sample large enough to
provide meaningful results.
Reducing this cut to 10\degrees\ resulted in a sample size too small to reveal
trends in the data, and intermediate cuts did not significantly change the
results.
% We also found that the anti-correlation between stellar mass and $b$ is
% negligable for stars cooler than 5000 K.
% For this reason, and because the rotational evolution of old, G stars does not
% follow a simple Skumanich-like spin-down law \citep{vansaders2016}, we
% restricted our sample to temperatures between \tmin\ and 5000 K, \ie\ K
% dwarfs.

\subsection{Establishing \vb\ as an age proxy}
\label{sec:mass-dependent-heating}

If lower-mass stars experience greater velocity changes when gravitationally
perturbed and are dynamically heated more efficiently than higher-mass stars,
velocity dispersion would be a function of {\it both} age and mass and cannot
be straightforwardly interpreted as an age proxy.
So, in order to establish whether \sigmavb\ can be used as an age proxy, we
searched for signs of mass-dependent heating within the \kepler\ field.

Mass-dependent dynamical heating has not been unambiguously observed in the
galactic disk because of the strong anti-correlation between stellar mass and
stellar age.
Less massive stars do indeed have larger velocity dispersions, however they
are also older on average.
This mass-age degeneracy is highly reduced in M dwarfs because their
main-sequence lifetimes are longer than the age of the Universe, however no
evidence for mass-dependent heating has been found in M dwarfs
\citep{faherty2009}.

To investigate whether mass-dependent heating could be acting on the \kepler\
sample, we selected late K and M dwarfs observed by both \kepler\ and \gaia,
whose MS lifetimes exceed around 11 Gyrs and are therefore representative of
the initial mass function.
We could not perform this analysis on the \mct\ sample, because only stars
with {\it detectable} rotation periods appear there.
Since lower-mass stars stay active for longer it is likely that there are more
old stars (with larger velocity dispersions) at low masses in this sample.
We selected all \kepler\ targets with dereddened \gaia\ \gcolor\ colors
greater than 1.2 (corresponding to an effective temperature $\lesssim$
4800 K) and absolute \gaia\ $G$-band magnitudes $<$ 4.
We also eliminated visual binaries by removing stars above a 6th order
polynomial, fit to the MS on the \gaia\ CMD.
We then applied the quality cuts described in section \ref{sec:the_data}.
To search for evidence of mass-dependent heating we calculated the (\vb)
velocity dispersion of stars in effective temperature bins.
Sigma clipping was performed at the 3$\sigma$ level to remove high velocity
outliers before calculating the standard deviation of stars in each bin.

Figure \ref{fig:vb_vs_teff} shows velocity and velocity dispersion
as a function of effective temperature (calculated by transforming dereddened
\gaia\ colors using equation \ref{eq:curtis}).
Velocity dispersion very slightly {\it decreases} with decreasing temperature,
the opposite of the trend expected for mass-dependent heating, however the
slope is only inconsistent with zero at the 1.3 $\sigma$ level.
This trend may be due to a selection bias: cooler stars are fainter and
therefore typically closer, with smaller heights above the galactic plane and
smaller velocities.
The essential point is that we do not find evidence for mass-dependent heating
acting on stars in the \kepler\ field, meaning \sigmavb\ {\it can} be used as
an age proxy.
We performed the same analysis on the 537 stars in this sample with RVs using
{\it vertical} velocity (\vz) and found the slope of the velocity
dispersion-temperature relation was consistent with zero.

Since we used \vb, not \vz\ in our analysis, it is possible that an
anti-correlation between \teff\ and galactic latitude, $b$, could influence
our results.
Stars at higher latitudes have additional velocity components in the ${\bf x}$
and ${\bf y}$ directions, which could increase \vb\ but not \vz.
Again however, since the relationship between \sigmavb\ and \teff\ is
positively, not negatively correlated for cool stars in the \kepler\ field,
this effect is probably too small to influence our results.

\begin{figure}
  \caption{
      Top: Stellar velocity (\vb) as a function of \teff\ for
      \kepler\ K and M dwarfs.
Vertical lines indicate different \teff-groupings used to calculate velocity
    dispersion.
Pink stars were not included in velocity dispersion calculations as they were
    either removed as outliers during a sigma clipping process, or they lie at
    the sparcely populated, extremely cool end of the temperature range.
    Velocity dispersion and \teff\ are slightly positively correlated, likely
    due to a brightness-related selection bias, indicating that mass-dependent
    heating does not significantly affect low-mass stars in the \kepler\
    field.
}
  \centering
    \includegraphics[width=1\textwidth]{vb_vs_teff}
\label{fig:vb_vs_teff}
\end{figure}

% Figure \ref{fig:vb_vs_teff} indicates that mass-dependent heating does not
% strongly affect the \mct\ sample of rotating \kepler\ stars.
% For this reason, we assume that age difference is the major cause of velocity
% dispersion differences between groups.
% In other words, (\vb) velocity dispersion is a reliable age proxy for the
% \mct\ sample.
