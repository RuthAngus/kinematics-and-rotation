% 270 words
% Limit: 500 words

\subsection{The data}
\label{sec:the_data}

We used the publicly available \kepler-\gaia\ DR2 crossmatched
catalog\footnote{Available at gaia-kepler.fun} to combine the \mct\ catalog of
stellar rotation periods, measured from \kepler\ light curves, with the \gaia\
DR2 catalog.
Reddening and extinction from dust was calculated for each star using the
Bayestar dust map implemented in the {\tt dustmaps} {\it Python} package
\citep{green2018}, and {\tt astropy} \citep{astropy2013}.
We estimated effective temperatures from dereddened \Gaia\ \gcolor\ color,
using an 8th-order polynomial relation calibrated using .... stars
\racomment{ask Jason for details}.
\begin{equation}
    \mathrm{T_{eff}} = 8960 -4802C + 1931C^2 -2446C^3 + 2669C^4 - 1324C^5 +
    301C^6 - 26C^7,
% 8959.8112335205078, -4801.5566310882568, 1931.4756631851196,
%           -2445.9980716705322, 2669.0248055458069, -1324.0671020746231,
%           301.13205924630165, -25.923997443169355]
\label{eq:curtis}
\end{equation}
where C is \gaia\ \gcolor.

% \begin{figure}
%   \caption{
% A \gaia\ color magnitude diagram showing the \citet{mcquillan2014} sample with
%     extinction-corrected magnitudes, colored by rotation period.
% We excluded photometric binaries and subgiants from our analysis by removing
% stars above the two dashed lines.
% % The rotation periods of binaries and subgiants do not follow a Skumanich-like
% % braking law.
% % The rotation period gradient across the main sequence is visible by eye in
% %     this figure: young, rapidly rotating stars are located below the old,
% %     slowly rotating stars.
% % since we rely on a simple scaling
% % between rotation period and age to make the argument that the rotation period
% % is not caused by incorrect period measurements.
% }
%   \centering
%     \includegraphics[width=1\textwidth]{CMD_cuts}
% \label{fig:CMD_cuts}
% \end{figure}
% To explore the age of this stellar population from a rotation standpoint, it
% was first necessary to remove visual binaries and subgiants from the sample.
% The rotational evolution of these two types of stars is generally different to
% that of single stars which more usually follow a Skumanich-like spin-down law.
% Tidal and magnetic interactions between the two components of a binary system
% can influence the rotation periods of both stars, and the expanding envelopes
% of subgiants drive rapid spin-down through conservation of angular momentum.
We removed visual binaries and subgiants from the sample by applying cuts to
the color-magnitude diagram (CMD), shown in figure \ref{fig:age_gradient}.
We fit a 6th-order polynomial to the main sequence and raised it by 0.27 dex
to approximate the division between single stars and visual binaries and
removed all stars above this line from the sample.
We also removed subgiants by eliminating stars brighter than 6th magnitude in
\gaia\ G-band.
\begin{figure}
  \caption{
Dereddened MS \kepler\ stars with \mct\ rotation periods on the \gaia\ CMD.
We excluded visual binaries by removing stars above the dashed line.
    Points are colored by their gyrochronal
    age, according to the
    \citet{angus2019} gyrochronology relation.
    A general age gradient is visible across the main sequence.
}
  \centering
    \includegraphics[width=1\textwidth]{age_gradient_combo}
\label{fig:age_gradient}
\end{figure}

The {\tt Pyia} \citep{price-whelan_2018} and {\tt astropy} \citep{astropy2013,
astropy2018} {\it Python} packages were used to calculate stellar velocities.
{\tt Pyia} has built-in functionality for calculating velocity samples from
the full \gaia\ uncertainty covariance matrix via Monte Carlo sampling.
It therefore not only incorporates uncertainties on the \gaia\ positions
parallaxes and proper motions, it also accounts for the {\it covariance}
between these properties.
We removed stars with negative parallaxes, parallax signal-to-noise ratios
less than 10, stars fainter than 16th magnitude, stars with absolute \vb\
uncertainties greater than 1 \kms\, and stars with galactic latitudes greater
than 15\degrees\ from the sample.
Because \vb\ is only a close approximation to \vz\ at low galactic latitudes,
and because latitude is correlated with stellar mass (lower mass stars are
older and tend to be preferentially located at high $b$), we found that
including stars with $b > 15$\degrees\ led to a larger number of \vb\ outliers
at low stellar mass.
15\degrees\ was found to be an adequate compromise between maintaining a close
relationship between \vb\ and \vz\ and keeping a large enough sample size to
provide meaningful results as we found that reducing this to 10\degrees did
not significantly affect our results.
We also found that the anti-correlation between stellar mass and $b$ is
negligable for stars cooler than 5000 K.
For this reason, and because the rotational evolution of old, G stars does not
follow a simple Skumanich-like spin-down law \citep{vansaders2016}, we
restricted our sample to temperatures between \tmin\ and 5000 K, \ie\ K
dwarfs.

% \begin{figure}
%   \caption{
% the rotation periods of stars in the \citet{mcquillan2014} sample vs.
% effective temperature.
% black points are dwarf stars identified as non-photometric binaries, hotter
% than 4800 k.
% blue circle points are non-photometric binary dwarfs, cooler than 4800 k, with
% a rotation period and \gaia\ color indicating they are older than 1.1 gyr.
% orange squares are stars that with rotation periods that fall just below the
% gap: they have rotation-ages between 0.5 and 1.1 gyrs.
% green triangles are stars with rotation periods faster than the main envelop
% of stars.
% these are probably binaries whose rotation periods are synchronized to their
% orbits and have been spun-up via tidal interactions.
% }
%   \centering
%     \includegraphics[width=1\textwidth]{period_teff}
% \label{fig:period_teff}
% \end{figure}
