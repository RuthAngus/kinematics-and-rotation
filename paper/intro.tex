% Word limit: 500
\section{Introduction}

\subsection{Gyrochronology}

Stars with significant convective envelopes ($\lesssim$ 1.3 M$_\odot$) have
strong magnetic fields and slowly lose angular momentum via magnetic braking
\citep[\eg][]{schatzman1962, weber1967, skumanich1972, kawaler1988,
pinsonneault1989}.
Although stars are typically born with random rotation periods, ranging from 1
to 10 days, observations of young open clusters reveal that their rotation
periods converge onto a unique sequence by $\sim$500-700 million years
\citep[\eg][]{irwin2009, gallet2013}.
After this time, the rotation period of a star is thought to be determined, to
first order, by its color and age alone.
This is the principle behind gyrochronology, the method of inferring a
star’s age from its rotation period \citep[\eg][]{barnes2003, barnes2007,
barnes2010, meibom2011, meibom2015}.
% It is well established that magnetized stellar winds cause the rotation
% periods of FGKM dwarfs to increase over time \citep[\eg][]{schatzman1962,
% weber1967, skumanich1972, kawaler1988, pinsonneault1989}, and, once fully
% calibrated, the relationship between period, mass and age can be used to
% age-date stars via gyrochronology \citep{barnes2003, barnes2007, barnes2010,
% meibom2011, meibom2015}.
However, new photometric rotation periods made available by the \kepler\
\citep{borucki2010} and \ktwo\ \citep{howell2014} missions
\citep[\eg][]{mcquillan2014, garcia2014, douglas2017, rebull2017, meibom2011,
meibom2015, curtis2019} have revealed that rotational evolution is more
complicated than previously thought.
For example, the M dwarfs in the $\sim$ 650 Myr Praesepe cluster spin more
slowly than the G dwarfs.
In theory this is because lower-mass stars have deeper convections zones which
generate stronger magnetic fields and more efficient magnetic braking.
However, in the 1.1 Gyr NGC 6811 cluster, late-K dwarfs rotate at the {\it
same} rate as early-K dwarfs \citep{curtis2019}.
In other words, convection zone depth cannot be the only variable that affects
stellar spin-down rate.
New semi-empirical models that vary the rate of angular momentum
redistribution in the interiors of stars are able to reproduce this flattened
period-color relation \citep{spada2019}.
These models suggest that mass and age-dependent angular momentum transport
between the cores and envelopes of stars has a significant impact on their
surface rotation rates.
Another example of unexpected rotational evolution is seen in old field stars
which appear to rotate more rapidly than classical gyrochronology models
predict \citep{angus2015, vansaders2016, vansaders2018, metcalfe2019}.
A mass-dependent modification to the classical \prot\ $\propto
t^{\frac{1}{2}}$ spin-down law \citep{skumanich1972} is required to reproduce
these observations.
To fit magnetic braking models to these data, a cessation of magnetic braking
is required after stars reach a Rossby number (Ro; the ratio of rotation
period to convective turnover time) of around 2 \citep{vansaders2016,
vansaders2018}.

The rotational evolution of stars is clearly a complicated process and, to
fully calibrate the gyrochronology relations we need a large sample of
reliable ages for stars spanning a range of ages and masses.
% however relying on open clusters (which are typically young) and asteroseismic
% stars (which are typically G-type stars or more massive) to calibrate
% gyrochronology relations limits the mass and age coverage of the calibration
% sample.
In this paper, we use the velocity dispersions of field stars to qualitatively
explore the rotational evolution of GKM dwarfs, and show that kinematics could
provide a gyrochronology calibration sample.

\subsection{Using kinematics as an age proxy}

Stars are thought to be born in the thin disk of the Milky Way (MW), orbiting
the galaxy with a low out-of-plane, or vertical, velocity ($W$, or \vz),
just like the star-forming molecular gas observed in the disk today
\citep[\eg][]{stark1989, stark2005, aumer2009, martig2014, aumer2016}.
On average, the vertical velocities of stars increase over time
\citep[\eg][]{nordstrom2004, holmberg2007, holmberg2009, aumer2009,
casagrande2011}.
Although the cause of dynamical heating is not well understood, interactions
with giant molecular clouds, spiral arms and the galactic bar are thought to
play an important role \citep[see][for a review of secular evolution in the
MW]{sellwood2014}.
Although the velocity of any individual star will only provide a weak age
constraint, the velocity {\it dispersion} of a {\it group} of stars can
indicate whether, on average, that group is old or young relative to other
groups.
In this work we compare the velocity dispersions of groups of field stars in
the Galactic thin disk to ascertain which groups are older and which younger
and draw conclusions based on the implied relative ages.
% The age-velocity dispersion reations (AVRs) are still actively being

% Vertical action is thought to be a better age indicator than vertical
% velocity \citep{ting2019}, although still only a weak age indicator for
% individual stars \citep{beane2018}, however both vertical action ($J_z$) and
% vertical
Although {\it vertical} velocity, \vz, is a well-established age proxy, it can
only be calculated with full 6-dimensional position and velocity information.
Unfortunately most field stars with measured rotation periods do not have
radial velocity (RV) measurements because they are relatively faint \kepler\
targets ($\sim$12th-16th magnitudes).
For this reason, we used velocity in the direction of galactic latitude, \vb,
to approximate \vz.
The \kepler\ field is positioned at low galactic latitude
(b=$\sim$5-20\degrees), so \vb\ is a close (although imperfect, see section
\ref{sec:results}) approximation to \vz.
Because we use \vb\ rather than \vz\, we cannot calculate absolute kinematic
ages using an age-velocity dispersion relation (AVR).
However, regardless of direction, velocity dispersion is expected to
monotonically increase over time, and can therefore be used to {\it rank}
groups of stars by age.

This paper is laid out as follows: in section \ref{sec:method} we describe our
sample selection process and the methods used to calculated stellar
velocities.
We also establish that \vb\ velocity dispersion, \sigmavb, can be used as an
age proxy by demonstrating that neither mass-dependent heating nor the
selection function seems to strongly affect on our sample.
In section \ref{sec:results} we use kinematics to investigate the
relationship between rotation period, age and color/\teff\ in the field and
interpret our results in section \ref{sec:discussion}.
% We show that the period-color relation of Praesepe is not applicable to
% old stars in section \ref{sec:age_cut}, and reveal the true shape of the
% period-color relations in section \ref{sec:the_reveal}.
% We discuss a possible connection with the rotation period gap in section
% \ref{sec:period_gap}.

% We also tested the \citet{angus2019} gyrochronology relation, a separable
% relation in color and age, calibrated using the period-color relation of
% Praesepe (in \gaia\ \gcolor\ color) and the period-age relation of Praesepe
% and the Sun.
% The large number of Praesepe members with precise rotation periods from the
% \ktwo\ mission \citep{douglas2017, rebull2017}, spanning spectral types F
% through early M, makes it a good cluster for calibrating the period-color
% relation of stars at 650 Myrs.
% However, although this relation accurately describes the rotation periods of F
% and G stars up to around 2.5 Gyr (the age of NGC 6819 -- the oldest cluster
% with available rotation periods), it over-predicts the rotation periods of K
% dwarfs in the 1.1 Gyr NGC 6811 cluster.
% Very few reliable age estimates exist for K dwarfs with rotation periods older
% than 1.1 Gyr, or between the ages of $\sim$ 800 Myr--1.1 Gyr so the rotational
% evolution of middle-aged K dwarfs is largely unknown.
% The oldest M dwarfs with reliable ages and rotation periods are members of
% Praesepe ($\sim$ 650 Myrs), so even {\it less} is known about their rotational
% evolution.

% We calculated gyrochronal ages of cool field dwarfs in the \kepler\ field
% using the \citep{angus2019} gyrochronology model, with de-reddened \Gaia\
% \gcolor\ color and rotation periods reported by \mct\ (we explain our data
% selection process in section \ref{sec:the_data}).
% These gyrochronal ages are shown on a \gaia\ color-magnitude diagram (CMD) in
% figure \ref{fig:age_gradient}.
% The stars with old gyrochronal ages, plotted in yellow hues, predominantly lie
% along the upper edge of the MS, where stellar evolution models predict old
% stars to be, however the majority of these `old' stars are bluer than \gcolor\
% $\sim$ 1.5 dex.
% This suggests that the \citet{angus2019} gyrochronology relation
% under-predicts the ages of low-mass stars.
% There is no reason to expect the oldest stars in this sample to be the bluer
% ones: M dwarfs are, on average, older than K dwarfs and are expected to remain
% active for longer, so should therefore have measurable rotation periods at
% older ages.
% Since the \citet{angus2019} gyrochronology model, which is based on the
% period-color relation of Praesepe, predicts the oldest stars in this sample to
% be K dwarfs, it is probably either under-predicting M dwarf ages or
% over-predicting K dwarf ages.
% % In other words, old M dwarfs rotate more rapidly or K dwarfs rotate more
% % slowly than the model predicts.
% In what follows, we use a population-based stellar age indicator, velocity
% dispersion, to investigate the gyrochronology relations at old ages in
% the field.
