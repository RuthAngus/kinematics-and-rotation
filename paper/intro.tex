% Word limit: 500
\section{Introduction}

\subsection{Gyrochronology}
The rotation periods of FGKM dwarfs increase with the square-root of time, and
this phenomenon was first observed several decades ago \citep{skumanich1972}.
This characteristic of main-sequence stars allows them to be dated via their
rotation periods, a practice known as gyrochronology.
This is convenient, since the ages of main-sequence stars are extremely
difficult, and at times impossible, to measure via the traditional age-dating
method of isochrone placement.
It has also been known for a long time that stars of the same age but
different masses have different rotation periods \racomment{(citation)}.
However, an underlying assumption behind many of the empirical gyrochronology
relations \citep[\eg][]{barnes2003, barnes2007, mamajek2008, meibom2011,
angus2015, angus2019} is that the relationships between rotation period and
color, and rotation period and age are {\it separable}.
In other words, the period-color relation is the same at all ages and the
period-age relation is the same at all colors.
It was recently shown that old stars appear to be too rapidly rotating for
their age, \citep{angus2015, vansaders2016, vansaders2018} and that a
mass-dependent modification to the classical \citet{skumanich1972} spin-down
law was needed in order to fit the data \citep{vansaders2016, vansaders2018}.
In addition, an even more recent analysis of middle-aged open clusters provide
further hints that the spin-down rate is mass-dependent \citep{curtis2019}.
Unfortunately, without rotation periods for members of old open clusters, or
any large group of old stars with precisely measured ages, it is extremely
difficult to calibrate the rotation-color relations at old ages.
For this reason, we turn to a population-based stellar age proxy: velocity
dispersion.

\subsection{Kinematics as an age proxy}

Stars are usually born in the thin disk of the Milky Way and orbit the center
of the galaxy with a low out-of-plane, or vertical, velocity ($W$, or $v_z$).
Young stars have relatively low vertical velocities, but gain angular momentum
in the vertical direction over time.
Although the cause of orbital heating is not well understood, interactions
with giant molecular clouds are thought to play an important role.

Stellar velocities have a long history of being used as an age proxy, with
several notable examples within stellar astronomy \citep[\eg][]{faherty2009,
west2011}.

A star's vertical {\it action} is a better indication of its age than its
velocity, because it is integrated over the Milky Way's potential, and
therefore position invariant -- \eg\ a star will have the same action at
periapsis and apoapsis.
In contrast, a star's velocity will be different at periapsis and apoapsis --
so in this sense, {\it actions} are the natural quantities to use as age
proxies.
However, actions can only be calculated will full 6-dimensional position and
velocity information, yet most stars with rotation periods do not have radial
velocity measurements because they are \kepler\ targets and most \kepler\
targets are relatively faint (between $\sim$11th and $~$18th magnitudes).
For this reason, we used an alternative age proxy: velocity in the direction
of galactic latitude, \vb.
Since these stars only have proper motions, parallaxes and positions, with no
radial velocites, we could not calculate full 3D velocities or actions.
However, since the \kepler\ field is at low galactic latitude, \vb is a close
approximation to $v_z$.

% The rotation periods of middle-aged FGKM stars are mostly determined by their
% mass, age and evolutionary stage, and this is due, for the most part, to
% angular momentum loss through magnetic braking.
% Although stars are born with a range of rotation periods, a steep dependence
% of angular momentum loss, $J$ angular frequency, $\omega$, $J \propto
% \omega^3$ causes stellar rotation periods to converge after a characteristic
% timescale.
% This timescale is on the order of a few tens of millions of years for F and G
% stars, but hundreds of millions of years for K and M stars.
% Once stars exceed this age, they steadily lose angular momentum via magnetized
% stellar winds that remove angular momentum from the star at a radius that is
% determined by the magnetic field strength.
% The Rossby number, the ratio of rotation period to convective turnover
% timescale, is linked to the strength of stellar magnetic dynamos.
% Large-scale polodial and toroidal stellar magnetic fields are thought to be
% generated by the $\alpha-\Omega$ mechanism, whereby convective motion of
% plasma and differential stellar rotation cause the winding up of magnetic
% field lines, generating a strong magnetic field \racomment{(citations)}.
% In general, the ratio between a star's rotation period and its characteristic
% timescale of convective motion (the Rossby number) determines the strength of
% the magnetic field \racomment{(citation)}.
% Stars that rotate quickly and have deep convection zones, therefore long
% convective turnover times, have small Rossby numbers and strong magnetic
% fields.
% Young M dwarfs typically have the strongest magnetic fields.
% Slowly rotating stars with shallow convective zones, such as old F and G
% dwarfs, have weak magnetic fields \racomment{(citations)}.
% Since magnetic field strength depends on stellar rotation period and mass via
% the Rossby number, it follows that angular momentum loss driven by magnetized
% stellar winds also depends on Rossby number.
