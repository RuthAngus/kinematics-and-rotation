% Word limit: 500
\section{Introduction}

\subsection{Gyrochronology}
It is well established that the rotation periods of FGKM dwarfs increase over
time \citep{skumanich1972}, however the exact nature and form of rotational
evolution is still being actively studied.
Once rotational evolution is well understood, rotation periods can be used to
precisely and accurately date stars via gyrochronology
\citep[\eg][]{kawaler1989, pinsonneault1989, barnes2003, barnes2007,
barnes2010, meibom2011, meibom2015}.
It is also well established that stars of the same age but different masses
have different rotation periods \citep[\eg][]{kraft1967, matt2012}, which is
thought to be caused by the variation in convection zone depth.
For example, lower mass stars with deeper convections zones are expected to
have stronger magnetic dynamos and lose angular momentum more rapidly.
However, as new rotation period measurements have become available, a more
complicated picture of stellar rotational evolution is emerging.
% However, an underlying assumption behind many empirical gyrochronology
% relations is that the relationships between rotation period and photometric
% color\footnote{As the directly observable quantity, color is often used as a
% mass proxy, and empirical gyrochronology relations are usually calibrated in
% color, rather than mass or effective temperature.},
% and rotation period and age are {\it separable}, meaning that the
% period-color relation is the same at all ages and the period-age relation is
% the same at all colors \citep[\eg][]{barnes2003, barnes2007, mamajek2008,
% meibom2011, angus2015, angus2019}.
% Not all gyrochronology models follow this form, particularly those based on
% stellar structure and evolution theory or with a Rossby number, rather than
% rotation period dependence \citep[\eg][]{barnes2010, matt2012,
% vansaders2016]}.
For example, it was recently shown that old field stars rotate more rapidly
than models predicted \citep{angus2015, vansaders2016, vansaders2018,
metcalfe2019}, and a mass-dependent modification to the classical \prot\
$\propto t^{\frac{1}{2}}$ spin-down law \citep{skumanich1972} was required to
reproduce the data \citep{vansaders2016, vansaders2018}.
Even more recent analyses of the 1.1 Gyr open cluster, NGC 6811, provides
further indication that the exponent of the period-age relation is mass and
time-dependent \citep{curtis2019 spada2019}.
This cluster has a flattened relationship between rotation period and
color: the G dwarfs rotate at the same rate as the K dwarfs.
New theoretical models that vary the rate of magnetic braking and the
redistribution of angular momentum in the interior of stars are able to
successfully reproduce the flattened period-color relation seen in old open
clusters \citep{spada2019}.
With the advent of large time-series photometry surveys, providing large
numbers of precise rotation period measurements over the last few years, we
now know that the rotational evolution of stars is a complicated process.
However, relying on open clusters (which are typically young) and
asteroseismic stars (which are typically G-type stars or more massive) to
calibrate gyrochronology limits the mass and age coverage of the calibration
sample.

In this paper, we tested the \citet{angus2019} gyrochronology relation, a
separable relation in color and age, calibrated using the period-color
relation of Praesepe (in \gaia\ \gcolor\ color) and the period-age relation of
Praesepe and the Sun.
The large number of Praesepe members with precise rotation periods from the
\ktwo\ mission \citep{douglas2017, rebull2017}, spanning spectral types F
through early M, makes it a good cluster for calibrating the period-color
relation of stars at 650 Myrs.
However, although this relation accurately describes the rotation periods of F
and G stars up to around 2.5 Gyr (the age of NGC 6819 -- the oldest cluster
with available rotation periods), it over-predicts the rotation periods of K
dwarfs in the 1.1 Gyr NGC 6811 cluster.
Very few reliable age estimates exist for K dwarfs with rotation periods older
than 1.1 Gyr, or between the ages of $\sim$ 800 Myr--1.1 Gyr so the rotational
evolution of middle-aged K dwarfs is largely unknown.
The oldest M dwarfs with reliable ages and rotation periods are members of
Praesepe ($\sim$ 650 Myrs), so even {\it less} is known about their rotational
evolution.

We calculated gyrochronal ages of cool field dwarfs in the \kepler\ field
using the \citep{angus2019} gyrochronology model, with de-reddened \Gaia\
\gcolor\ color and rotation periods reported by \mct\ (we explain our data
selection process in section \ref{sec:the_data}).
These gyrochronal ages are shown on a \gaia\ color-magnitude diagram (CMD) in
figure \ref{fig:age_gradient}.
The stars with old gyrochronal ages, plotted in yellow hues, predominantly lie
along the upper edge of the MS, where stellar evolution models predict old
stars to be, however the majority of these `old' stars are bluer than \gcolor\
$\sim$ 1.5 dex.
This provides the first hint that the \citet{angus2019} gyrochronology
relation under-predicts the ages of low-mass stars.
There is no reason to expect the oldest stars in this sample to be the bluer
ones: M dwarfs are, on average, older than K dwarfs and are expected to remain
active for longer, so should therefore have measurable rotation periods at
older ages.
Since the \citet{angus2019} gyrochronology model, which is based on the
period-color relation of Praesepe, predicts the oldest stars in this sample to
be K dwarfs, it is probably either under-predicting M dwarf ages or
over-predicting K dwarf ages.
% In other words, old M dwarfs rotate more rapidly or K dwarfs rotate more
% slowly than the model predicts.
In what follows, we use a population-based stellar age indicator, velocity
dispersion, to investigate the gyrochronology relations at old ages in
the field.

\subsection{Kinematics as an age proxy}

Stars are thought to be born in the thin disk of the Milky Way (MW), orbiting
the galaxy with a low out-of-plane, or vertical, velocity ($W$, or \vz),
just like the star-forming molecular gas observed in the disk today
\citep[\eg][]{stark1989, stark2005, aumer2009, martig2014, aumer2016}.
On average, the vertical velocities of stars increase over time
\citep[\eg][]{nordstrom2004, holmberg2007, holmberg2009, aumer2009,
casagrande2011}.
Although the cause of dynamical heating is not well understood, interactions
with giant molecular clouds, spiral arms and the galactic bar are thought to
play an important role \citep[see][for a review of secular evolution in the
MW]{sellwood2014}.
Although the velocity of any individual star will only provide a weak age
constraint, the velocity dispersion of a group of stars can indicate whether,
on average, that group is old or young relative to other groups.
In this work we compare the velocity dispersions of groups of stars to
ascertain which groups are older and which younger and draw conclusions based
on the implied relative ages of populations.
% The age-velocity dispersion relations (AVRs) are still actively being

% Vertical action is thought to be a better age indicator than vertical
% velocity \citep{ting2019}, although still only a weak age indicator for
% individual stars \citep{beane2018}, however both vertical action ($J_z$) and
% vertical
Vertical velocity, \vz, can only be calculated with full 6-dimensional
position and velocity information.
Unfortunately, most stars with measured rotation periods do not have radial
velocity (RV) measurements because they are relatively faint \kepler\ targets
($\sim$11th-18th magnitudes).
For this reason, we used velocity in the direction of galactic latitude, \vb.
The \kepler\ field is positioned at low galactic latitude (b=5-20\degrees), so
\vb\ is a close (although imperfect -- see section \ref{sec:results})
approximation to \vz.
Because we use \vb\ rather than \vz\, it is difficult to directly compare
gyrochronal ages with kinematic ones using an age-velocity dispersion relation
(AVR).
However, regardless of direction, velocity dispersion is expected to
monotonically increase over time, and can therefore be used effectively to
{\it rank} groups of stars by age.

This paper is laid out as follows: in section \ref{sec:method} we describe our
sample selection process and the methods used to calculated stellar
velocities.
We also establish that \vb\ velocity dispersion, \sigmavb, can be used as an
age proxy by demonstrating that neither mass-dependent heating nor the
selection function is likely to have a strong affect on our sample.
In section \ref{sec:results} we use stellar kinematics to investigate the
relationship between rotation period, age and color/\teff\ in the field.
We show that the period-color relation of Praesepe is not applicable to
old stars in section \ref{sec:age_cut}, and reveal the true shape of the
period-color relations in section \ref{sec:the_reveal}.
We discuss a possible connection with the rotation period gap in section
\ref{sec:period_gap}.
