% Word limit: 500
\section{Introduction}

\subsection{Gyrochronology}
It is well established that the rotation periods of FGKM dwarfs increase over
time \citep{skumanich1972}.
This characteristic of main-sequence (MS) stars allows them to be dated using
their rotation periods, via gyrochronology \citep[\eg][]{kawaler1989,
pinsonneault1989, barnes2003, barnes2007, barnes2010, meibom2011, meibom2015},
which is particularly useful since the ages of MS stars are difficult to
measure via isochrone placement.
It is also well established that stars of the same age but different masses
have different rotation periods \citep[\eg][]{kraft1967, matt2012}, thought to
be caused by the deeper convective zones, and therefore stronger magnetic
dynamos (and more efficient magnetic braking) in lower-mass stars.
However, an underlying assumption behind many empirical gyrochronology
relations is that the relationships between rotation period and photometric
color\footnote{As the directly observable quantity, color is often used as a
mass proxy, and empirical gyrochronology relations are usually calibrated in
color, rather than mass or effective temperature.},
and rotation period and age are {\it separable}, meaning that the
period-color relation is the same at all ages and the period-age relation is
the same at all colors \citep[\eg][]{barnes2003, barnes2007, mamajek2008,
meibom2011, angus2015, angus2019}.
Not all gyrochronology models follow this form, particularly those motivated
based on stellar structure and evolution theory or with a Rossby number,
rather than rotation period dependence \citep[\eg][]{barnes2010, matt2012,
vansaders2016]}.
It was recently shown that old field stars rotate more rapidly than a simple,
separable gyrochronology relation would predict \citep{angus2015,
vansaders2016, vansaders2018, metcalfe2019}, and that a mass-dependent
modification to the classical \prot\ $\propto t^{\frac{1}{2}}$ spin-down law
\citep{skumanich1972} is required to reproduce the data \citep{vansaders2016,
vansaders2018}.
An even more recent analysis of the 1.1 Gyr open cluster, NGC 6811, provides
further indication that the exponent of the period-age relation is
mass and time-dependent \citep{curtis2019}.
This cluster has a flattened relationship between rotation period and color
and the G dwarfs rotate at the same rate as the K dwarfs.

In this paper, we tested the \citet{angus2019} gyrochronology relation, a
separable relation, calibrated using the period-color relation of Praesepe (in
\gaia\ \gcolor\ color) and the period-age relation of Praesepe and the Sun.
The large number of Praesepe members with precise rotation periods from the
\ktwo\ mission \citep{douglas2017, rebull2017}, spanning spectral types F
through early M, makes it a good cluster for calibrating the period-color
relation of stars at 650 Myrs.
This relation accurately describes the rotation periods of F and G stars up to
around 2.5 Gyr (the age of NGC 6819 -- the oldest cluster with available
rotation periods), but over-predicts the rotation periods of K dwarfs in the
1.1 Gyr NGC 6811 cluster.
Very few reliable age estimates exist for K dwarfs with rotation periods older
than 1.1 Gyr, or between the ages of $\sim$ 800 Myr--1.1 Gyr so the rotational
evolution of middle-aged K dwarfs is difficult to characterize\footnote{The
oldest M dwarfs with reliable ages and rotation periods are members of
Praesepe ($\sim$ 650 Myrs), so even less is known about their rotational
evolution.}.

The gyrochronal ages of the stars in our sample were calculated with the
\citep{angus2019} gyrochronology model, using de-reddened \Gaia\ \gcolor\
color and rotation periods reported by \mct.
These rotational ages are shown on the CMD in figure \ref{fig:age_gradient}.
The stars with old rotational ages, plotted in yellow hues, predominantly lie
along the upper edge of the MS, where stellar evolution models predict old
stars to be, however the majority of stars with old rotational ages are bluer
than \gcolor\ $\sim$ 1.5 dex.
This provides a first hint that the \citet{angus2019} gyrochronology relation
under-predicts the ages of low-mass stars.
There is no reason to expect the oldest stars in this sample to be the bluer
ones: M dwarfs are, on average, older than K dwarfs and are expected to remain
active for longer, so should therefore have measurable rotation periods at
older ages.
Since the \citet{angus2019} gyrochronology model, which is based on the
period-color relation of Praesepe, does not predict old ages for any M dwarfs,
it is probably under-predicting the ages of some of these low-mass stars.
It is therefore likely that old M dwarfs rotate more rapidly than this
Praesepe-based model predicts.
In what follows we use a population-based stellar age indicator, velocity
dispersion, to further investigate the gyrochronology relations at old ages in
the field.

\subsection{Kinematics as an age proxy}

Stars are thought to be born in the thin disk of the Milky Way (MW), orbiting
the center of the galaxy with a low out-of-plane, or vertical, velocity ($W$,
or $v_z$), just like the star-forming molecular gas observed in the disk today
\citep[\eg][]{stark1989, stark2005, aumer2009, martig2014, aumer2016}.
Young stars have relatively small vertical velocities, but gain momentum in
the vertical direction over time \citep[\eg][]{nordstrom2004, holmberg2007,
holmberg2009, aumer2009, casagrande2011}.
Although the cause of orbital heating is not well understood, interactions
with giant molecular clouds, spiral arms and the galactic bar are thought to
play an important role \citep[see][for a review of secular evolution in the
MW]{sellwood2014}.
Although the velocity of any individual star will only provide a weak age
constraint, the velocity dispersion of a group of stars can indicate whether,
on average, that group is old or young relative to other groups.
In this work we compare the velocity dispersions of groups of stars to
ascertain which groups are older and which younger and draw conclusions based
on the implied relative ages of populations.
The age-velocity dispersion relations (AVRs) are still actively being
calibrated, so it is difficult to directly compare gyrochronal ages with
kinematic ones.
However, regardless of the exact relation between velocity dispersion and
stellar age, it is expected to be a monotonic relationship, therefore velocity
dispersion can be used effectively to {\it rank} groups of stars by age.

Vertical action is thought to be a better age indicator than vertical velocity
\citep{ting2019}, although still only a weak age indicator for individual
stars \citep{beane2018}, however both vertical action ($J_z$) and vertical
velocity (\vz/W) can only be calculated with full 6-dimensional position and
velocity information.
Unfortunately, most stars with measured rotation periods do not have radial
velocity (RV) measurements because they are relatively faint \kepler\ targets
($\sim$11th-18th magnitudes).
For this reason, we used velocity in the direction of galactic latitude, \vb.
The \kepler\ field is positioned at low galactic latitude (b=5-20\degrees), so
\vb\ is a close (although imperfect -- see section \ref{sec:results})
approximation to \vz.


This paper is laid out as follows: in section \ref{sec:the_data} we describe
our sample selection process and the methods used to calculated stellar
velocities.
In section \ref{sec:results} we use stellar kinematics to investigate the
relationship between rotation period, age and color/\teff\ in the field.
In section \ref{sec:mass-dependent-heating} we show that neither
mass-dependent heating nor the selection function is likely to have a strong
affect on our sample, and in section \ref{sec:period_gap} we discuss the
implications of our results.
