\section{Introduction}

\subsection{Gyrochronology}

Stars with significant convective envelopes ($\lesssim$ 1.3 M$_\odot$) have
strong magnetic fields and slowly lose angular momentum via magnetic braking
\citep[\eg][]{schatzman1962, weber1967, skumanich1972, kawaler1988,
pinsonneault1989}.
Although stars are born with random rotation periods, between 1 and 10 days,
observations of young open clusters reveal that their rotation periods
converge onto a unique sequence by $\sim$500-700 million years
\citep[\eg][]{irwin2009, gallet2013}.
After this time, the rotation period of a star is thought to be determined, to
first order, by its color and age alone.
This is the principle behind gyrochronology, the method of inferring a
star’s age from its rotation period \citep[\eg][]{barnes2003, barnes2007,
barnes2010, meibom2011, meibom2015}.
However, new photometric rotation periods made available by the \kepler\
\citep{borucki2010} and \ktwo\ \citep{howell2014} missions
\citep[\eg][]{mcquillan2014, garcia2014, douglas2017, rebull2017, meibom2011,
meibom2015, curtis2019} have revealed that rotational evolution is more
complicated than previously thought.
For example, the early-to-mid M dwarfs in the $\sim$ 650 Myr Praesepe cluster
spin more slowly than the G dwarfs; in theory because lower-mass stars have
deeper convective zones which generate stronger magnetic fields and more
efficient magnetic braking.
However, in the NGC 6811 cluster which is around 1.1 Gyr \citep{janes2011},
late-K dwarfs rotate at the {\it same rate} as early-K dwarfs
\citep{curtis2019}.
In other words, convection zone depth cannot be the only variable that affects
stellar spin-down rate.
New semi-empirical models that vary the rate of angular momentum
redistribution in the interiors of stars are able to reproduce this flattened
period--color relation \citep{spada2019}.
These models suggest that mass and age-dependent angular momentum transport
between the cores and envelopes of stars has a significant impact on their
surface rotation rates.
Another example of unexpected rotational evolution is seen in old field stars
which appear to rotate more rapidly than classical gyrochronology models
predict \citep{angus2015, vansaders2016, vansaders2018, metcalfe2019}.
A mass-dependent modification to the classical \prot\ $\propto
t^{\frac{1}{2}}$ spin-down law \citep{skumanich1972} is required to reproduce
these observations.
To fit magnetic braking models to these data, a cessation of magnetic braking
is required after stars reach a Rossby number (Ro; the ratio of rotation
period to convective turnover time) of around 2 \citep{vansaders2016,
vansaders2018}.

The rotational evolution of stars is clearly a complicated process and, to
fully calibrate the gyrochronology relations we need a large sample of
reliable ages for stars spanning a range of ages and masses.
In this paper, we use the velocity dispersions of field stars to qualitatively
explore the rotational evolution of GKM dwarfs, and show that kinematics could
provide a gyrochronology calibration sample.

\subsection{Using kinematics as an age proxy}

Stars are thought to be born in the thin disk of the Milky Way (MW), orbiting
the Galaxy with a low out-of-plane, or vertical, velocity (\vz),
just like the star-forming molecular gas observed in the disk today
\citep[\eg][]{stark1989, stark2005, aumer2009, martig2014, aumer2016}.
On average, the vertical velocities of older stars is observed to be larger
\citep[\eg][]{nordstrom2004, holmberg2007, holmberg2009, aumer2009,
casagrande2011}.
This is likely either a signature of dynamical heating, such as from
interactions with giant molecular clouds, spiral arms and the galactic bar
\citep[see][for a review of secular evolution in the MW]{sellwood2014}, or an
indication that stars formed dynamically ``hotter'' in the past
\citep[e.g.,][]{bird2013}.
In either case, the vertical velocity distribution is observed to depend
significantly on stellar age.
While the velocity of any individual star only provides a weak age constraint,
because its velocity depends on its current position in its orbit, the
velocity {\it dispersion} of a {\it group} of stars indicates whether that
group is old or young relative to other groups.
% Vertical velocity dispersion is thought to increase monotonically with age.
In this work, we compare the velocity dispersions of groups of field stars in
the Galactic thin disk to ascertain which groups are older and which younger,
and draw conclusions about the rotational evolution of stars based on their
implied relative ages.

Although {\it vertical} velocity, \vz, is an established age proxy, it can
only be calculated with full 6-dimensional position and velocity information.
In fact, with full 6D phase space and an assumed Galactic potential, it is
possible to calculate the dynamically-invariant vertical {\it action}, which may
be an even better age indicator \citep{beane2018, ting2019}.
Unfortunately, most field stars with measured rotation periods do not have
radial velocity (RV) measurements because they are relatively faint \kepler\
targets ($\sim$12th-16th magnitudes).
For this reason, we used velocity in the direction of galactic latitude, \vb,
as a proxy for \vz.
The \kepler\ field is positioned at low galactic latitude
(b=$\sim$5-20\degrees), so \vb\ is a close (although imperfect, see
appendix) approximation to \vz.
Because we use \vb\ rather than \vz\, we do not calculate absolute kinematic
ages using a published age--velocity dispersion relation (AVR), calibrated with
vertical velocity.
In future it may be possible to account for the differences between \vb\ and
\vz, or marginalize over missing RV measurements and the \kepler\ selection
function, in order to infer absolute ages.
Regardless of direction however, velocity dispersion is expected to
monotonically increase over time \citep[\eg][]{holmberg2009}, and can
therefore be used to {\it rank} groups of stars by age.

This paper is laid out as follows: in section \ref{sec:method} we describe our
sample selection process and the methods used to calculate stellar
velocities.
In section \ref{sec:results} we use kinematics to investigate the relationship
between stellar rotation period, age and color/\teff\ and interpret the
results.
We also examine the rotation period gap and the kinematics of synchronized
binaries.
In the appendix, we establish that \vb\ velocity dispersion, \sigmavb, can be
used as an age proxy by demonstrating that neither mass-dependent heating nor
the \kepler/\gaia\ selection function is observed to strongly affect our
sample.
