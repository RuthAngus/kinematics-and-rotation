% Word limit: 500
\section{Introduction}

\subsection{Gyrochronology}
The rotation periods of FGKM dwarfs increase with the square-root of time, and
this phenomenon was first observed several decades ago \citep{skumanich1972}.
This characteristic of main-sequence stars allows them to be dated via their
rotation periods, a practice known as gyrochronology.
This is convenient, since the ages of main-sequence stars are extremely
difficult, and at times impossible, to measure via the traditional age-dating
method of isochrone placement.
It has also been known for a long time that stars of the same age but
different masses have different rotation periods \racomment{(citation)}.
However, an underlying assumption behind many of the empirical gyrochronology
relations \citep[\eg][]{barnes2003, barnes2007, mamajek2008, meibom2011,
angus2015, angus2019} is that the relationships between rotation period and
color, and rotation period and age are {\it separable}.
In other words, the period-color relation is the same at all ages and the
period-age relation is the same at all colors.
It was recently shown that old stars appear to be too rapidly rotating for
their age, \citep{angus2015, vansaders2016, vansaders2018} and that a
mass-dependent modification to the classical \citet{skumanich1972} spin-down
law was needed in order to fit the data \citep{vansaders2016, vansaders2018}.
In addition, an even more recent analysis of middle-aged open clusters
provides further hints that the spin-down rate is mass-dependent
\citep{curtis2019}.
Rotation period measurements of the 1.1 Gyr open cluster NGC 6811 reveal a
flattened relationship between rotation period and color, where the G dwarfs
are rotating at the same rate as the K dwarfs.

Unfortunately, without rotation periods for members of old open clusters, or
any large group of old stars with precisely measured ages, it is extremely
difficult to calibrate the rotation-color relations at old ages.
For this reason, we turned to a population-based stellar age proxy: velocity
dispersion in order to investigate the period-temperature relations at old
ages.

\subsection{Kinematics as an age proxy}

Stars are thought to be born in the thin disk of the Milky Way, orbiting the
center of the galaxy with a low out-of-plane, or vertical, velocity ($W$, or
$v_z$).
Young stars have relatively small vertical velocities, but gain angular
momentum in the vertical direction over time.
Although the cause of orbital heating is not well understood, interactions
with giant molecular clouds are thought to play an important role.

Stellar velocities have a long history of being used as an age proxy, with
several notable examples within stellar astronomy \citep[\eg][]{faherty2009,
west2011}.

Vertical {\it actions} are better age indicators than velocities, because
actions are calculated by integrating angular momentum over the Milky Way's
potential, and are therefore position invariant -- \eg\ a star will have the
same action at periapsis and apoapsis.
In contrast, a star's velocity will be different at periapsis and apoapsis --
so in this sense, {\it actions} are the natural quantities to use as age
proxies.
However, actions can only be calculated will full 6-dimensional position and
velocity information, yet most stars with measured rotation periods do not
have radial velocity measurements because they are \kepler\ targets and most
\kepler\ targets are relatively faint (between $\sim$11th and $~$18th
magnitudes).
For this reason, we used an alternative age proxy: velocity in the direction
of galactic latitude, \vb.
Since these stars only have proper motions, parallaxes and positions, with no
radial velocites, we could not calculate full 3D velocities or actions.
However, since the \kepler\ field is at low galactic latitude, \vb is a close
approximation to $v_z$.

Although the velocity of any individual star will only provide a weak age
constraint, the velocity dispersion of a group of stars can indicate whether,
on average, that group is old or young relative to other groups.
In this work we compare the velocity dispersions of groups of stars to
ascertain which groups are older and which younger and draw conclusions based
on the implied relative ages of populations.
